\newpage
\chapter{数列极限}
\section{实数系的连续性}
\subsection{实数系}
\begin{formal}
    \begin{definition}[数集的封闭性的定义]\label{def:数集的封闭性的定义}
        设$S$为一个数集,若$\forall a,b\in S,a$和$b$进行某种运算后的结果仍在$S$中,则称$S$对这种运算是封闭的.
    \end{definition}
\end{formal}
\begin{formal}
    \begin{definition}[可公度的定义]\label{def:可公度的定义}
        设两条线段长度为$a,b$,则必定存在一条长度为$c$的线段使得$c\mid a,c\mid b$,则称$a,b$可公度.
    \end{definition}
\end{formal}
\begin{red}
\begin{remark}
    整数集$\mathbb{Z}$具有离散性,有理数集$\mathbb{Q}$具有稠密性(但存在“空隙”),实数集$\mathbb{R}$具有连续性,故$\mathbb{R}$又称为实数连续统.
\end{remark}
\end{red}
\subsection{最大数与最小数}
\begin{formal}
    \begin{definition}[最大最小数的定义]\label{def:最大最小数的定义}
        设$S\subset\mathbb{R},$若$\exists\xi\in S$使得$\forall x\in S,x\leqslant\xi$,则称$\xi$为$S$的最大数;若$\exists\eta\in S$使得$\forall x\in S,x\geqslant\eta$,则称$\eta$为$S$的最小数.
    \end{definition}
\end{formal}
\begin{formal}
    \begin{theorem}[最大最小数的存在性]\label{thm:最大最小数的存在性}
        设$S\subset\mathbb{R},$若$S$是非空有限集,则\cref{def:最大最小数的定义}中的最大数$\xi$与最小数$\eta$必定存在.若$S$为无限集,则最大数与最小数不一定存在.
    \end{theorem}
\end{formal}
\subsection{上确界与下确界}
\begin{formal}
    \begin{definition}[上界与下界的定义]\label{def:上界与下界的定义}
        设$\varnothing\neq S\subset\mathbb{R}$,若$\exists M\in\mathbb{R}$使得$\forall x\in S,x\leqslant M,$则称$M$为$S$的上界;若$\exists m \in\mathbb{R}$使得$\forall x\in S,x\geqslant m,$则称$m$为$S$的下界.
        
        当$S$既有上界又有下界时,称$S$有界.
    \end{definition}
\end{formal}
\begin{formal}
    \begin{definition}[有界的等价定义]\label{def:有界的等价定义}
        根据\cref{def:上界与下界的定义},$S$有界当且仅当$\exists M>0$使得$\forall x\in S,|x|\leqslant M$.
    \end{definition}
\end{formal}
\begin{formal}
    \begin{definition}[上确界与下确界的定义]\label{def:上确界与下确界的定义}
        设$\varnothing\neq S\subset\mathbb{R}$,在\cref{def:上界与下界的定义}基础上,$U$为全体上界组成的集合,显然$U$无最大数,但$U$有最小数$\beta$,称为$S$的上确界,记为$\beta=\sup S$;$L$为全体下界组成的集合,显然$L$无最小数,但$L$有最大数$\alpha$,称为$S$的下确界,记为$\alpha=\inf S$.
    \end{definition}
\end{formal}
\begin{formal}
    \begin{theorem}[上确界与下确界的性质]\label{thm:上确界与下确界的性质}
        \cref{def:上确界与下确界的定义}中的上确界$\beta=\sup S$满足:\begin{enumerate}[label={\textup{(\arabic*)}}]
            \item $\beta$是$S$的上界:$\forall x\in S,x\leqslant\beta$;
            \item $\beta$是$S$的最小上界:$\forall\varepsilon>0,\exists x\in S$使得$\beta-\varepsilon<x\leqslant\beta$;
        \end{enumerate}
        \cref{def:上确界与下确界的定义}中的下确界$\alpha=\inf S$满足:\begin{enumerate}[label={\textup{(\arabic*)}}]
            \item $\alpha$是$S$的下界:$\forall x\in S,x\geqslant\alpha$;
            \item $\alpha$是$S$的最大下界:$\forall\varepsilon>0,\exists x\in S$使得$\alpha\leqslant x<\alpha+\varepsilon$.
        \end{enumerate}
    \end{theorem}
\end{formal}
\begin{formal}
    \begin{theorem}[确界存在定理/实数系连续性定理]\label{thm:确界存在定理/实数系连续性定理}
        非空有上界的实数集必有上确界,非空有下界的实数集必有下确界.
    \end{theorem}
    \begin{Proof}
        $\forall x\in\mathbb{R},x=\left[\,x\,\right]+\left\{x\right\},$其中\[
        \left\{x\right\}=0.a_1a_2a_3\cdots a_n\cdots
        \]注意到$0.a_1a_2\cdots a_p000\cdots=0.a_1a_2\cdots \left(a_p-1\right)999\cdots\left(a_p\neq 0\right),$因此避开使用后者.于是任一实数有这样的唯一表示方法\[
        S=\left\{
            a_0+0.a_1a_2\cdots a_n\cdots\mid a_0=\left[\,x\,\right],0.a_1a_2\cdots a_n\cdots=\left\{x\right\},x\in S
        \right\}
        \]设$S$有上界,取$S$中$a_0$的最大者$\alpha_0$并记\[
        S_0=\left\{
            x\in S\mid \left[\,x\,\right]=\alpha_0
        \right\}\Longrightarrow\forall x\in S,x\notin S_0,x<\alpha_0
        \]再取$S_0$中$a_1$的最大者为$\alpha_1$,作\[
        S_1=\left\{
            x\in S\mid a_1=\alpha_1
        \right\}\Longrightarrow\forall x\in S,x\notin S_1,x<\alpha_1+0.\alpha_1
        \]一般地,考虑$S_{n-1}$中$a_n$的最大者为$\alpha_n$,记\[
        S_n=\left\{
            x\in S_{n-1}\mid a_n=\alpha_n
        \right\}\Longrightarrow\forall x\in S,x\notin S_n,x<\alpha_n+0.\alpha_1\alpha_2\cdots\alpha_n
        \]于是得到一系列的非空数集$
        S\supset S_0\supset S_1\supset\cdots\supset S_n\supset\cdots
        $和一系列数字\[
        \alpha_0\in\mathbb{Z},\alpha_1,\alpha_2,\cdots,\alpha_n,\cdots\in\left\{
            0,1,2,3,4,5,6,7,8,9
        \right\}
        \]作$\beta=\alpha_0+0.\alpha_1\alpha_2\cdots\alpha_n\cdots$.这样完成了构造,下证$\beta$确实是$S$的上确界.

        一方面,$\forall x\in S,$则要么$\exists n_0\in\mathbb{N},x\notin S_{n_0},$要么$\forall n\in\mathbb{N},x\in S_n.$第一种情况下\[
        x<\alpha_0+0.\alpha_1\alpha_2\cdots\alpha_{n_0}\leqslant \beta
        \]第二种情况下$x=\beta$,于是$\forall x\in S,x\leqslant \beta$即$\beta$是$S$的上界.

        另一方面,$\forall \varepsilon >0,$取$n_0\mathbb{N}$使得$\displaystyle \frac{
            1
        }{10^{n_0}}<\varepsilon.$取$x_0\in S_{n_0}$则$x_0$与$\beta$的整数部分及前$n_0$位小数相同,于是\[
        \beta-x_0\leqslant\frac{1}{10^{n_0}}<\varepsilon
        \]即$\forall \varepsilon>0,\exists x_0>\beta-\varepsilon$即$\beta-\varepsilon$不是$S$的上界,故$\beta$是$S$的上确界.

        同理,对于$S$有下界的情况,可证$S$有下确界.
    \end{Proof}
\end{formal}
\begin{red}
    \begin{remark}
        上述证明中,$\beta$的表示可能与约定相悖即\[
        \left\{\beta\right\}=0.\alpha_1\alpha_2\cdots999\cdots
        \]不过这并不影响,只要关心存在性即可.
    \end{remark}
\end{red}
\begin{formal}
    \begin{theorem}[确界的唯一性]\label{thm:确界的唯一性}
        非空有界数集的上确界与下确界是唯一的.
    \end{theorem}
    \begin{Proof}
        设$\sup S$等于$A$和$B$且$A<B$.取$\displaystyle \varepsilon=\frac{B-A}{2}>0$,考虑$B$即$\exists x\in S$是的$x>B-\varepsilon>A$,这与$A$是$S$的上确界矛盾.同理可证下确界的唯一性.
    \end{Proof}
\end{formal}
\begin{brown}
    \begin{example}
        证明$\displaystyle T=\left\{
            x\mid x\in\mathbb{Q},x>0,x^2<2
        \right\}$在$\mathbb{Q}$内无上确界.
    \end{example}
    \begin{Proof}
        考虑反证法,设上界$\displaystyle \sup T=\frac{n}{m}\left(m,n\in\mathbb{N}^+,\gcd{m}{n}=1\right)$,注意到$1.4^2<2<1.5^2$于是\[
        1<\left(
            \frac{n}{m}
        \right)^2<3
        \]并且$\sqrt{2}\notin \mathbb{Q}.$
        
        于是一方面$\displaystyle 1<\left(\frac{n}{m}\right)^2<2$,我们尝试寻找一个$r>0$使得$\displaystyle \frac{n}{m}+r\in T.$记$\displaystyle 2-\frac{n^2}{m^2}=t\in\left(0,1\right)$,作$\displaystyle r=\frac{n}{6m}t,$显然$\displaystyle 0<\frac{n}{m}+r\in\mathbb{Q}.$因为$\displaystyle r^2=\frac{n^2t^2}{36m^2}<\frac{t}{18}$并且$\displaystyle \frac{2nr}{m}=\frac{n^2t}{3m^2}<\frac{2t}{3}$于是\[
        \left(
            \frac{n}{m}+r
        \right)-2=r^2+2\frac{nr}{m}-t<\frac{t}{18}+\frac{2t}{3}-t<0\Longrightarrow
        \frac{n}{m}<\frac{n}{m}+r\in T\Longrightarrow \frac{n}{m}\neq\sup T
        \]

        另一方面$\displaystyle 2<\left(\frac{n}{m}\right)^2<3$,我们尝试寻找一个$r>0$使得$\displaystyle \frac{n}{m}-r\in T.$记$\displaystyle \frac{n^2}{m^2}-2=t\in\left(0,1\right)$,作$\displaystyle r=\frac{n}{6m}t,$显然$\displaystyle 0<\frac{n}{m}-r\in\mathbb{Q}.$因为$\displaystyle \frac{2nr}{m}=\frac{n^2t}{3m^2}<t$于是\[
        \left(
            \frac{n}{m}-r
        \right)^2-2=r^2+2\frac{nr}{m}-t>0\Longrightarrow
        \frac{n}{m}-r\in T\Longrightarrow \frac{n}{m}\neq\sup T
        \]综上所述,$\displaystyle T$在$\mathbb{Q}$内无上确界.
    \end{Proof}
\end{brown}
\newpage
\section{数列极限}
\subsection{数列与数列极限}
\begin{formal}
    \begin{definition}[数列的定义]\label{def:数列的定义}
        数列是一串按照正整数编了号的数\[
        \left\{x_n\right\}:x_1,x_2,x_3,\cdots,x_n,\cdots
        \]其中$x_n$称为该数列的通项.
    \end{definition}
\end{formal}
\begin{formal}
    \begin{definition}[数列极限的定义]\label{def:数列极限的定义}
        设数列$\left\{x_n\right\}$,$a$为一个实常数.若$\forall \varepsilon >0,\exists N\in\mathbb{N}^+$使得$n>N$时\[
        \left|
            x_n-a
        \right|<\varepsilon
        \]恒成立,则称$a$为数列$\left\{x_n\right\}$的极限(数列$\left\{x_n\right\}$收敛于$a$),记作\[
        \lim_{n\to\infty}x_n=a\text{或}x_n\to a\left(n\to\infty\right)
        \]若不存在这样的常数$a$,则称数列$\left\{x_n\right\}$发散.
    \end{definition}
\end{formal}
\begin{formal}
    \begin{definition}[邻域的概念]\label{def:邻域的概念}
        平面直角坐标系中$x$轴上以$x_0$为中心,以$\delta>0$为半径的开区间$\left(
            x_0-\delta,x_0+\delta
        \right)$称为$x_0$的$\delta$邻域,记作\[
        U\left(x_0,\delta\right)=\left(
            x_0-\delta,x_0+\delta
        \right)
        \]特别地,去掉$x_0$的$\delta$邻域称为$x_0$的$\delta$去心邻域,记作\[
        \mathring{U}\left(x_0,\delta\right)=\left(
            x_0-\delta,x_0\right)\cup\left(
                x_0,x_0+\delta
            \right)
        \]
    \end{definition}
\end{formal}
\begin{red}
    \begin{remark}
        数列的收敛与否、收敛到何处与它的前有限项无关.
    \end{remark}
\end{red}
\begin{brown}
    \begin{example}
        证明$\displaystyle \left\{
            \frac{n}{n+3}
        \right\}$收敛于$1$.
    \end{example}
    \begin{Proof}
        即\[
        \left|
            \frac{n}{n+3}-1
        \right|<\varepsilon\Longrightarrow n>\frac{3}{\varepsilon}-3
        \]取$N=\left\lceil\displaystyle\frac{3}{\varepsilon} \right\rceil -3$即可.
    \end{Proof}
\end{brown}
\begin{formal}
    \begin{definition}[无穷小量的定义]\label{def:无穷小量的定义}
        极限为$0$的数列称为无穷小量.
    \end{definition}
\end{formal}
\begin{red}
    \begin{remark}
        无穷小量是变量而不是一个非常小的量.特别地,数列\[
        0,0,0,\cdots,0,\cdots
        \]是一个特殊的无穷小量.
    \end{remark}
\end{red}
\begin{green}
\begin{corollary}[无穷小量的推出]\label{cor:无穷小量的推出}
    根据\cref{def:数列极限的定义}可以构造出$\left\{
        x_n-a
    \right\}$这样一个无穷小量.
\end{corollary}
\end{green}
\begin{brown}
    \begin{example}
        $\left\{q^n\right\}\left(0<\left|q\right|<1\right)$是无穷小量.
    \end{example}
\end{brown}
\begin{brown}
    \begin{example}
        证明:$\displaystyle \lim_{n\to\infty}\sqrt[n]{n}=1.$
    \end{example}
    \begin{Proof}
        令$\displaystyle \sqrt[n]{n}=1+y_n$则\[
        n=\left(1+y_n\right)^n=1+ny_n+\frac{n\left(n-1\right)}{2}y_n^2+\cdots+y_n^n>1+\frac{n\left(n-1\right)}{2}y_n^2
        \]即\[
        \left|\sqrt[n]{n}-1\right|=\left|y_n\right|<\sqrt{\frac{2}{n}}
        \]于是对于$\forall \varepsilon>0,$取$N=\left\lceil\displaystyle\frac{2}{\varepsilon^2}\right\rceil$即可.
    \end{Proof}
    事实上\[
    \lim_{n\to\infty}\sqrt[n]{x}=1
    \]其中$x$可以是任意正实数,也可以是$n$的$n>k\in\mathbb{N}^+$次幂.
\end{brown}
\begin{brown}
    \begin{example}
        证明:若$\displaystyle \lim_{n\to\infty}a_n=a$,则\[
        \lim_{n\to\infty}\frac{a_1+a_2+\cdots+a_n}{n}=a
        \]
    \end{example}
    \begin{Proof}
        先假设$a=0$,于是$\forall \varepsilon>0,\exists N_1$使得$n>N_1$时\[
        \left|a_n\right|<\frac{\varepsilon}{2}
        \]于是\begin{align*}
            \left|\frac{
            a_1+a_2+\cdots+a_n
            }{n}\right|&\leqslant\left|\frac{
                a_1+a_2+\cdots+a_{N_1}
            }{n}\right|+\left|\frac{
                a_{N_1+1}+\cdots+a_n
            }{n}\right|\\
            &<\left|
                \frac{
                    a_1+a_2+\cdots+a_{N_1}
                }{n}
            \right|+\frac{\varepsilon}{2}
        \end{align*}此时再取定$N>N_1$使得$n>N$时\[
        \left|
            \frac{
                a_1+a_2+\cdots+a_{N_1}
            }{n}
        \right|<\frac{\varepsilon}{2}
        \]于是\[
        \left|
            \frac{
                a_1+a_2+\cdots+a_n
            }{n}
        \right|<\varepsilon
        \]

        $a\neq 0$时,根据\cref{cor:无穷小量的推出}知\[
        \lim_{n\to\infty}\frac{a_1+a_2+\cdots+a_n}{n}-a=\lim_{n\to\infty}\frac{\left(a_1-a\right)+\left(a_2-a\right)+\cdots+\left(a_n-a\right)}{n}=0
        \]即\[
        \lim_{n\to\infty}\frac{a_1+a_2+\cdots+a_n}{n}=a\qedhere
        \]

        也可使用\cref{thm:Stolz定理}
    \end{Proof}
\end{brown}
\subsection{数列极限的性质}
\begin{formal}
    \begin{definition}[数列极限的等价表述]\label{def:数列极限的等价表述}
        数列$\left\{x_n\right\}$极限为$a$的等价表述为\[
        \forall \varepsilon>0,\exists N,\forall n>N:\left|x_n-a\right|<\varepsilon
        \]
    \end{definition}
\end{formal}
\begin{formal}
    \begin{theorem}[极限唯一]\label{thm:极限唯一}
        收敛数列的极限唯一.
    \end{theorem}
    \begin{Proof}
        设$\left\{x_n\right\}$的极限为$a$和$b$,则有\cref{def:数列极限的等价表述}知\[
        \forall \varepsilon>0,\exists N_1,\forall n>N_1:\left|x_n-a\right|<\frac{\varepsilon}{2};\forall \varepsilon>0,\exists N_2,\forall n>N_2:\left|x_n-b\right|<\frac{\varepsilon}{2}
        \]取$N=\max\,\left\{N_1,N_2\right\}$则$\forall n>N$:\[
        \left|a-b\right|=\left|
            a-x_n+x_n-b
        \right|\leqslant \left|
            x_n-a
        \right|+\left|
            x_n-b
        \right|<\varepsilon\to 0
        \]于是$a=b.$
    \end{Proof}
\end{formal}
\begin{formal}
    \begin{definition}[有界数列的定义]\label{def:有界数列的定义}
        对于数列$\left\{x_n\right\}$,若存在$M\in\mathbb{R}$使得\[
        x_n\leqslant M,\forall n\in\mathbb{N}^+
        \]则称$M$为数列$\left\{x_n\right\}$的一个上界.若存在$m\in\mathbb{R}$使得\[
        x_n\geqslant m,\forall n\in\mathbb{N}^+
        \]则称$m$为数列$\left\{x_n\right\}$的一个下界.若上界与下界均存在,则称数列$\left\{x_n\right\}$有界.
    \end{definition}
\end{formal}
\begin{formal}
    \begin{definition}[有界数列的等价定义]\label{def:有界数列的等价定义}
        \cref{def:有界数列的定义}等价于:存在$0<X\in\mathbb{R}$使得\[
        \left|x_n\right|\leqslant X,\forall n\in\mathbb{N}^+
        \]
    \end{definition}
\end{formal}
\begin{formal}
    \begin{theorem}[收敛与有界]\label{thm:收敛与有界}
        收敛数列必定有界,有界数列不一定收敛.
    \end{theorem}
    \begin{Proof}
        设数列$\left\{x_n\right\}$收敛于$a$,取定$\varepsilon=1$则$\exists N,\forall n>N:\left|x_n-a\right|<1$则\[
        a-1<x_n<a+1
        \]取$M=\max\,\left\{x_1,x_2,\cdots,x_N,a+1\right\},m=\min\,\left\{
            x_1,x_2,\cdots,x_N,a-1
        \right\}$即可.

        $\left\{
            \left(-1\right)^n
        \right\}$显然有界但不收敛.
    \end{Proof}
\end{formal}
\begin{formal}
    \begin{theorem}[数列的保序性]\label{thm:数列的保序性}
        设$\left\{x_n\right\}$收敛于$a$,$\left\{y_n\right\}$收敛于$b$且$a<b$,则$\exists N\in\mathbb{N}^+$使得$n>N$时\[
        x_n<y_n
        \]
    \end{theorem}
    \begin{Proof}
        取$\displaystyle \varepsilon=\frac{b-a}{2}$于是\[
        \exists N_1,\forall n>N_1:\left|x_n-a\right|<\frac{b-a}{2}\Longrightarrow x_n<a+\frac{b-a}{2}=\frac{a+b}{2}
        \]同理\[
        \exists N_2,\forall n>N_2:\left|y_n-b\right|<\frac{b-a}{2}\Longrightarrow y_n>b-\frac{b-a}{2}=\frac{a+b}{2}
        \]于是取$N=\max\,\left\{N_1,N_2\right\}$即可.
    \end{Proof}
\end{formal}
\begin{green}
    \begin{corollary}[保序性推论]\label{cor:保序性推论}
        由\cref{thm:数列的保序性}知\begin{enumerate}[label={\textup{(\arabic*)}}]
            \item 若$\displaystyle \lim_{n\to\infty}y_n>0$,则$\exists N$使得$n>N$时\[
            y_n>\frac{b}{2}>0
            \]
            \item 若$\displaystyle \lim_{n\to\infty}y_n<0$,则$\exists N$使得$n>N$时\[
             y_n<\frac{b}{2}<0
             \]
        \end{enumerate}
    \end{corollary}
\end{green}
\begin{red}
    \begin{remark}
        \cref{cor:保序性推论}说明当$\left\{x_n\right\}$极限不为$0$,$n$充分大时$x_n$不能充分接近$0$.
    \end{remark}
\end{red}
\begin{red}
    \begin{remark}
        \cref{thm:数列的保序性}的逆命题并不成立如$\displaystyle \left\{\frac{1}{n}\right\}$和$\displaystyle \left\{\frac{2}{n}\right\}$.事实上,只能有\cref{cor:保序性逆命题改进}.
    \end{remark}
\end{red}
\begin{green}
    \begin{corollary}[保序性逆命题改进]\label{cor:保序性逆命题改进}
        若$\displaystyle \lim_{n\to\infty}x_n=a,\lim_{n\to\infty}y_n=b$且$\exists N\in\mathbb{N}^+$使得$n>N$时$x_n\leqslant y_n$,则$a\leqslant b.$
    \end{corollary}
\end{green}
\begin{formal}
    \begin{criterion}[夹逼准则]\label{cri:夹逼准则}
        设数列$\left\{x_n\right\},\left\{y_n\right\},\left\{z_n\right\}$从某一项开始有\[
        x_n\leqslant y_n\leqslant z_n,n>N_0
        \]且$\displaystyle \lim_{n\to\infty}x_n=\lim_{n\to\infty}z_n=a$则$\displaystyle\lim_{n\to\infty}y_n=a.$
    \end{criterion}
    \begin{Proof}
        $\forall\varepsilon>0$,$\exists N_1,\forall n>N_1:\left|x_n-a\right|<\varepsilon\Longrightarrow a-\varepsilon<x_n$;$\exists N_2,\forall n>N_2:\left|z_n-a\right|<\varepsilon\Longrightarrow z_n<a+\varepsilon$.则取$N=\max\,\left\{N_0,N_1,N_2\right\},\forall n>N:$\[
        a-\varepsilon<x_n\leqslant y_n\leqslant z_n<a+\varepsilon
        \]即\[
        \left|y_n-a\right|<\varepsilon\qedhere
        \]
    \end{Proof}
\end{formal}
\begin{brown}
    \begin{example}
        证明:\[
        \lim_{n\to\infty}\left(
            a_1^n+a_2^n+\cdots+a_p^n
        \right)^{\frac{1}{n}}=\max_{1\leqslant i\leqslant p}\,\left\{a_i\right\}
        \]其中$a_i\geqslant0\left(i=1,2,\cdots,p\right).$
    \end{example}
\end{brown}
\subsection{数列极限的四则远算}
\begin{formal}
    \begin{theorem}[极限的四则运算]\label{thm:极限的四则运算}
        设$\displaystyle \lim_{n\to\infty}x_n=a,\lim_{n\to\infty}y_n=b$,则\begin{enumerate}[label={\textup{(\arabic*)}}]
            \item $\lim_{n\to\infty}\left(\alpha x_n+\beta y_n\right)=\alpha a+\beta b$($\alpha,\beta$为常数)
            \item $\displaystyle \lim_{n\to\infty}\left(x_ny_n\right)=ab$
            \item $\displaystyle \lim_{n\to\infty}\left(
                \frac{x_n}{y_n}
            \right)=\frac{a}{b}\left(b\neq 0\right)$
        \end{enumerate}
    \end{theorem}
    \begin{Proof}
        首先$\exists X>0$使得$\forall n\in\mathbb{N}^+,\left|x_n\right|\leqslant X$且$\forall \varepsilon>0,\exists N_1,\forall n>N_1:\left|x_n-a\right|<\varepsilon ,\exists N_2,\forall n>N_2:\left|y_n-b\right|<\varepsilon$,则取$N=\max\,\left\{N_1,N_2\right\},\forall n>N:$\begin{align*}
        \left|
            \left(\alpha x_n+\beta y_n\right)-\left(\alpha a+\beta b\right)
        \right|&=\left|
            \alpha\left(x_n-a\right)+\beta\left(y_n-b\right)
        \right|\\
        &\leqslant \left|
            \alpha\left(x_n-a\right)
        \right|+\left|
            \beta\left(y_n-b\right)
        \right|\\
        &\leqslant\left|\alpha\right|\left|x_n-a\right|+\left|\beta\right|\left|y_n-b\right|\\
        &<\left(\left|\alpha\right|+\left|\beta\right|\right)\varepsilon
        \end{align*}\begin{align*}
            \left|x_ny_n-ab\right|&=\left|
                x_n\left(y_n-b\right)+b\left(x_n-a\right)
            \right|\\
            &<\left|X+\left|b\right|\right|\varepsilon
        \end{align*}
        根据\cref{cor:保序性推论}知$\displaystyle \exists N_0,\forall n>N_0:\left|y_n\right|>\frac{\left|b\right|}{2},$取$N=\max\,\left\{
            N_0,N_1,N_2
        \right\},\forall n>N:$\begin{align*}
        \left|
            \frac{x_n}{y_n}-\frac{a}{b}
        \right|&=\left|
            \frac{b\left(x_n-a\right)-a\left(y_n-b\right)}{y_nb}
        \right|\\
        &<\frac{2\left(\left|a\right|+\left|b\right|\right)}{b^2}\varepsilon
        \end{align*}
    \end{Proof}
\end{formal}
\begin{green}
    \begin{corollary}[开方运算]\label{cor:开方运算}
        设$\displaystyle x_n\geqslant 0,\lim_{n\to\infty}x_n=a\geqslant 0\Longrightarrow\lim_{n\to\infty}\sqrt{x_n}=\sqrt{a}.$
    \end{corollary}
\end{green}
\begin{red}
    \begin{remark}
        \cref{thm:极限的四则运算}只对有限个数列正确,对无穷个或不定数个不总成立.

        事实上,涉及无穷的时候总要小心而慎重.
    \end{remark}
\end{red}
\newpage
\section{无穷大量}
\subsection{无穷大量}
\begin{formal}
    \begin{definition}[无穷大量的定义]\label{def:无穷大量的定义}
        若$\forall G>0,\exists N,n>N:$\[
        \left|x_n\right|>G
        \]则称数列$\left\{x_n\right\}$为无穷大量.记作\[
        \lim_{n\to\infty}x_n=\infty
        \]
    \end{definition}
\end{formal}
\begin{formal}
    \begin{definition}[定号无穷大量]\label{def:定号无穷大量}
        若\cref{def:无穷大量的定义}中无穷大量从某一项开始均是正数,则称数列$\left\{x_n\right\}$为正无穷大量,记作\[
        \lim_{n\to\infty}x_n=+\infty
        \]若无穷大量从某一项开始均是负数,则称数列$\left\{x_n\right\}$为负无穷大量,记作\[
        \lim_{n\to\infty}x_n=-\infty
        \]二者统称为定号无穷大量,否则称为不定号无穷大量.
    \end{definition}
\end{formal}
\begin{formal}
    \begin{theorem}[无穷大与无穷小]\label{thm:无穷大与无穷小}
        $\left\{x_n\right\}$是无穷大量当且仅当$\displaystyle \left\{\frac{1}{x_n}\right\}$是无穷小量.
    \end{theorem}
    \begin{Proof}
        先证必要性,$\forall \varepsilon>0,$取$\displaystyle G=\frac{1}{\varepsilon}>0$则$\displaystyle\exists N,\forall n>N:\left|x_n\right|>G=\frac{1}{\varepsilon}$从而\[
        \left|\frac{1}{x_n}\right|<\varepsilon
        \]即$\displaystyle\left\{\frac{1}{x_n}\right\}$是无穷小量.

        同理可证充分性.
    \end{Proof}
\end{formal}
\begin{formal}
    \begin{theorem}[无穷大量与非零量]\label{thm:无穷大量与非零量}
        设$\left\{x_n\right\}$为无穷大量,若$n>N_0$时$\left|y_n\right|\geqslant\delta>0,$则$\left\{x_ny_n\right\}$为无穷大量.
    \end{theorem}
\end{formal}
\begin{brown}
    \begin{example}
        $\displaystyle \left\{\frac{n}{\sin n}\right\}$为无穷大量因为\[
        \left|
            \frac{1}{\sin n}
        \right|\geqslant 1
        \]
    \end{example}
\end{brown}
\begin{formal}
    \begin{theorem}[无穷小量与有界量]\label{thm:无穷小量与有界量}
        设$\left\{x_n\right\}$为无穷小量,$\left|y_n\right|\leqslant M$,则$\left\{x_ny_n\right\}$为无穷小量.
    \end{theorem}
\end{formal}
\begin{formal}
    \begin{theorem}[无穷大量与有界量]\label{thm:无穷大量与有界量}
        设$\left\{x_n\right\}$为无穷大量,$\displaystyle\lim_{n\to\infty}y_n=b\neq 0$,则$\displaystyle\left\{x_ny_n\right\},\left\{\frac{x_n}{y_n}\right\}$都是无穷大量.
    \end{theorem}
\end{formal}
\begin{brown}
    \begin{example}
        $\left\{n\arctan n\right\}$为无穷大量因为\[
        \lim_{n\to\infty}\arctan n=\frac{\pi}{2}\neq 0
        \]
    \end{example}
\end{brown}
\begin{brown}
    \begin{example}
        设$k,l\in\mathbb{N}^+,a_0\neq 0,b_0\neq0$,则\[
        \lim_{n\to\infty}\frac{
            a_0n^k+a_1n^{k-1}+\cdots+a_{k-1}n+a_k
        }{
            b_0n^l+b_1n^{l-1}+\cdots+b_{l-1}n+b_l
        }=\begin{cases*}
            0&,$k<l$\\
            \cfrac{a_0}{b_0}&,$k=l$\\
            \infty&,$k>l$
        \end{cases*}
        \]
    \end{example}
\end{brown}
\begin{formal}
    \begin{proposition}[无穷大的运算]\label{prop:无穷大的运算}下面是关于无穷大量的运算

        \begin{enumerate}[label={\textup{(\arabic*)}}]
            \item $\left(+\infty\right)+\left(+\infty\right)=+\infty,\left(+\infty\right)-\left(-\infty\right)=+\infty$
            \item $\infty\pm$有界量$=\infty$
            \item $\left(+\infty\right)\cdot\left(+\infty\right)=+\infty,\left(+\infty\right)\cdot\left(-\infty\right)=-\infty,\left(-\infty\right)\cdot\left(-\infty\right)=+\infty$
        \end{enumerate}
    \end{proposition}
\end{formal}
\subsection{待定型}
\begin{formal}
    \begin{definition}[待定型的定义]\label{def:待定型的定义}
        除\cref{prop:无穷大的运算}外\begin{enumerate}[label={\textup{(\arabic*)}}]
            \item $\left(+\infty\right)-\left(+\infty\right)$
            \item $\left(+\infty\right)+\left(-\infty\right)$
            \item $\left(\infty\right)\pm\left(\infty\right)$
            \item $0\cdot\infty$
            \item $\displaystyle \frac{0}{0}$
            \item $\displaystyle \frac{\infty}{\infty}$
        \end{enumerate}都称为待定型.
    \end{definition}
\end{formal}
\begin{formal}
    \begin{definition}[数列单调性的定义]\label{def:数列单调性的定义}
        若数列$\left\{x_n\right\}$满足:\begin{enumerate}[label={\textup{(\arabic*)}}]
            \item \[
            x_n\leqslant x_{n+1},n=1,2,3,\cdots
            \]则称$\left\{x_n\right\}$为单调增加数列
            \item \[
            x_n<x_{n+1},n=1,2,3,\cdots
            \]则称$\left\{x_n\right\}$为严格单调增加数列
            \item \[
            x_n\geqslant x_{n+1},n=1,2,3,\cdots
            \]则称$\left\{x_n\right\}$为单调减少数列
            \item \[
            x_n>x_{n+1},n=1,2,3,\cdots
            \]则称$\left\{x_n\right\}$为严格单调减少数列
        \end{enumerate}
    \end{definition}
\end{formal}
\begin{formal}
    \begin{theorem}[Stolz定理]\label{thm:Stolz定理}
        设$\left\{y_n\right\}$是严格单调增加的正无穷大量,且\[
        \lim_{n\to\infty}   \frac{
            x_n-x_{n-1}
        }{y_n-y_{n-1}}=a\left(
            -\infty\leqslant a\leqslant +\infty
        \right)
        \]则\[
        \lim_{n\to\infty}\frac{x_n}{y_n}=a
        \]
    \end{theorem}
    \begin{Proof}
        设$a=0$即$\displaystyle \lim_{n\to\infty}\frac{
            x_n-x_{n-1}
        }{y_n-y_{n-1}}$则$\forall\varepsilon>0,\exists N_1,\forall n>N_1:$\[
        \left|
            \frac{
                x_n-x_{n-1}
            }{y_n-y_{n-1}}
        \right|<\varepsilon\Longrightarrow \left|x_n-x_{n-1}\right|<\varepsilon\left(y_n-y_{n-1}\right)
        \]于是$\left|x_n-x_{N_1}\right|\leqslant\varepsilon\left(y_n-y_{N_1}\right)$即\[
        \left|
            \frac{x_n}{y_n}-\frac{x_{N_1}}{y_n}
        \right|\leqslant\varepsilon\left(1-\frac{y_{N_1}}{y_n}\right)<\varepsilon
        \]于是\[
        \left|
            \frac{x_n}{y_n}
        \right|<\varepsilon+\left|
            \frac{x_{N_1}}{y_{n}}
        \right|
        \]现取$N>N_1$充分大使得$\forall n>N:\displaystyle \left|\frac{x_{N_1}}{y_n}\right|<\varepsilon$即\[
        \left|\frac{x_n}{y_n}\right|<2\varepsilon\Longrightarrow \lim_{n\to\infty}\frac{x_n}{y_n}=0=a
        \]

        $a\neq0$时,考虑$x_n'=x_n-ay_n$显然\[
        \lim_{n\to\infty}\frac{
            x_n'-x_{n-1}'
        }{
            y_n-y_{n-1}
        }=\lim_{n\to\infty}\left(
            \frac{x_n-x_{n-1}}{y_n-y_{n-1}}-a
        \right)=0
        \]于是\[
        \lim_{n\to\infty}\frac{x_n'}{y_n}=0\Longrightarrow \lim_{n\to\infty}\frac{x_n}{y_n}=a
        \]

        $a=+\infty$时即$\displaystyle \forall G>0,\exists N,\forall n>N:\frac{x_n-x_{n-1}}{y_n-y_{n-1}}>G$,取$G=1\Longrightarrow x_n-x_{n-1}>y_n-y_{n-1}>0\Longrightarrow\left\{x_n\right\}$严格单调递增且是正无穷大量,于是\[
        \lim_{n\to\infty}\frac{y_n-y_{n-1}}{x_n-x_{n-1}}=0\Longrightarrow \lim_{n\to\infty}\frac{y_n}{x_n}=0\Longrightarrow \lim_{n\to\infty}\frac{x_n}{y_n}=+\infty
        \]

        类似可证$a=-\infty$的情况.
    \end{Proof}
\end{formal}
\begin{brown}
    \begin{example}
        $k\in\mathbb{N}^+$,则
        \begin{align}
            \lim_{n\to\infty}\frac{1^k+2^k+\cdots+n^k}{n^{k+1}}&=\lim_{n\to\infty}\frac{n^k}{n^{k+1}-\left(n-1\right)^{k+1}}\\
            &=\lim_{n\to\infty}\frac{n^k}{
                C^1_{k+1}n^{k}+C^2_{k+1}n^{k-1}+\cdots
            }\\
            &=\frac{1}{k+1}
        \end{align}
    \end{example}
\end{brown}
\begin{red}
    \begin{remark}
        \cref{thm:Stolz定理}不能由$\infty\Longrightarrow\infty$如$x_n=\left(-1\right)^nn,y_n=n$,也不能由不存在推得不存在如$x_n=1-2+3-4+\cdots+\left(-1\right)^{n-1}n,y_n=n^2.$
    \end{remark}
\end{red}
\newpage
\section{收敛准则}
\begin{red}
    \begin{remark}
        考虑两个问题:\begin{enumerate}[label={\textup{(\arabic*)}}]
            \item 有界数列加上什么样的条件能够保证其收敛?
            \item 有界数列不加上任何条件能够给出什么样的结论?
        \end{enumerate}
    \end{remark}
\end{red}
\subsection{单调有界数列收敛定理}
\begin{formal}
    \begin{theorem}[单调收敛定理]\label{thm:单调收敛定理}
        单调有界数列必定收敛.
    \end{theorem}
    \begin{Proof}
        不妨设$\left\{x_n\right\}$单调递增且有上界,设$\beta=\sup\left\{x\mid x\in\left\{x_n\right\}\right\}$,由\cref{thm:上确界与下确界的性质}知$\forall \varepsilon>0:$\[
        \left|
            \beta-x_n
        \right|<\varepsilon
        \]于是可知$\displaystyle\lim_{n\to\infty}x_n=\beta$.
    \end{Proof}
\end{formal}
\begin{red}
    \begin{remark}
        这是充分条件,不是必要条件即并非所有收敛数列都是单调数列.
    \end{remark}
\end{red}
\begin{brown}
    \begin{example}
        设$\displaystyle x_1>0,x_{n+1}=1+\frac{x_n}{1+x_n},n=1,2,\cdots.$证明$\left\{x_n\right\}$收敛并求极限.
    \end{example}
    \begin{Proof}
        利用数学归纳法明显得到单调且有界,根据\cref{thm:单调收敛定理}知数列$\left\{x_n\right\}$收敛,两边同时求极限即得$\displaystyle \lim_{n\to\infty}x_n=\frac{1+\sqrt{5}}{2}.$
    \end{Proof}
\end{brown}
\begin{brown}
    \begin{example}
        设$0<x_1<1,x_{n+1}=x_n\left(1-x_n\right),n=1,2,3,\cdots.$证明$\left\{x_n\right\}$收敛并求极限.
    \end{example}
    \begin{Proof}
        易知单调减少且有下界,根据\cref{thm:单调收敛定理}知数列$\left\{x_n\right\}$收敛,两边同时求极限即得$\displaystyle \lim_{n\to\infty}x_n=0.$

        同时,根据\cref{thm:Stolz定理}也可求得\begin{align*}
        \lim_{n\to\infty}nx_n&=\lim_{n\to\infty}\frac{n}{\frac{1}{x_n}}\\
        &=\lim_{n\to\infty}\frac{1}{\cfrac{1}{x_{n+1}}-\cfrac{1}{x_n}}\\
        &=\lim_{n\to\infty}\frac{x_nx_{n+1}}{x_n-x_{n+1}}\\
        &=\lim_{n\to\infty}\frac{x_n^2\left(1-x_n\right)}{x_n^2}\\
        &=1\qedhere
        \end{align*}
    \end{Proof}
\end{brown}
\subsection{$\pi$与$\rme$}
\begin{brown}
    \begin{example}\label{ex:1}
        单位圆内接正$n$边形的半周长为\[
        L_n=n\sin\frac{180^\circ}{n}
        \]证明数列$\left\{L_n\right\}$收敛.
    \end{example}
    \begin{Proof}
        考虑单调性,令$\displaystyle t=\frac{180^\circ}{n\left(n+1\right)}$,则当$n\geqslant 3$时$nt\leqslant 45^\circ$.则\begin{align*}
            \tan nt=\frac{
                \tan\left(n-1\right)t+\tan t
            }{
                1-\tan\left(n-1\right)t\tan t
            }\geqslant\tan\left(n-1\right)t+\tan t\geqslant n\tan t
        \end{align*}于是\begin{align*}
            \sin\left(n+1\right)t=\sin nt\cos t+\cos nt\sin t=\sin nt\cos t\left(1+\frac{\tan t}{\tan nt}\right)\leqslant\frac{n+1}{n}\sin nt
        \end{align*}于是$n\geqslant 3$时\[
        L_n=n\sin\frac{180^\circ}{n}\leqslant\left(n+1\right)\sin\frac{180^\circ}{n+1}=L_{n+1}
        \]另一方面,单位圆内接正$n$边形的面积\[
        S_n=n\sin\frac{180^\circ}{n}\cos\frac{180^\circ}{n}<4
        \]故当$n\geqslant3$时\[
        L_n=n\sin\frac{180^\circ}{n}<\frac{4}{\cos\cfrac{180^\circ}{n}}\leqslant\frac{4}{\cos60^\circ}=8
        \]则由\cref{thm:单调收敛定理}知$\left\{L_n\right\}$收敛.定义\[
        \lim_{n\to\infty}n\sin\frac{180^\circ}{n}=\pi\qedhere
        \]
    \end{Proof}
    同时,外切正$n$边形的面积\[
    S'_n=n\tan\frac{180^\circ}{n}
    \]二者夹逼即可求得单位圆面积$S=\pi$.
\end{brown}
\begin{red}
    \begin{remark}
        弧度制下,\cref{ex:1}亦作\[
        \lim_{n\to\infty}\frac{\sin\left(\cfrac{\pi}{n}\right)}{\cfrac{\pi}{n}}=1\Longrightarrow\lim_{x\to0}\frac{\sin x}{x}=1
        \]后者将是重要的极限之一.
    \end{remark}
\end{red}
\begin{brown}
    \begin{example}
        证明$\displaystyle\left\{x_n=\left(1+\frac{1}{n}\right)^n\right\}$单调增加,$\displaystyle\left\{y_n=\left(1+\frac{1}{n}\right)^{n+1}\right\}$单调减少并且二者极限相同.
    \end{example}
    \begin{Proof}
        根据\cref{thm:基本不等式链}可得\[
        x_n=\left(1+\frac{1}{n}\right)^n\cdot1\leqslant\left[
            \frac{n\left(1+\frac{1}{n}\right)+1}{n+1}
        \right]^{n+1}=x_{n+1}
        \]以及\[
        \frac{1}{y_n}=\left(\frac{n}{n+1}\right)^{n+1}\cdot1\leqslant\left[
\frac{
    \left(n+1\right)\cfrac{n}{n+1}+1
}{n+2}
        \right]^{n+2}=\frac{1}{y_{n+1}}
        \]于是$\left\{x_n\right\}$单调增加,$\left\{y_n\right\}$单调减少.明显$2=x_1\leqslant x_n<y_n\leqslant y_1=4$并且二者显然具有相同的极限,记作\[
        \rme=\lim_{n\to\infty}\left(1+\frac{1}{n}\right)^n\qedhere
        \]
    \end{Proof}
\end{brown}
\begin{brown}
    \begin{example}\label{ex:2}
        设$\displaystyle a_n=1+\frac{1}{2^p}+\frac{1}{3^p}+\cdots+\frac{1}{n^p}$,求证:$p>1,\left\{a_n\right\}$收敛;$0,p\leqslant 1,\left\{a_n\right\}$发散.
    \end{example}
    \begin{Proof}
        $p>1$时记$\displaystyle \frac{1}{2^{p-1}}=r\Longrightarrow0<r<1$,于是\begin{align*}
            &\frac{1}{2^p}+\frac{1}{3^p}<\frac{1}{2^{p-1}}=r\\
            &\frac{1}{4^p}+\frac{1}{5^p}+\frac{1}{6^p}+\frac{1}{7^p}<\frac{1}{2^{2p-2}}=r^2\\
            &\cdots\cdots\\
            &\frac{1}{2^{kp}}+\frac{1}{\left(2^k+1\right)^p}+\cdots+\frac{1}{\left(2^{k+1}-1\right)^p}<r^k
        \end{align*}于是\[
        a_n\leqslant a_{2^n-1}<1+r+r^2+\cdots+r^{n-1}<\frac{1}{1-r}
        \]于是$p>1$时$\left\{x_n\right\}$收敛.

        $p\geqslant 1$时,作\begin{align*}
            &\frac{1}{2^p}>\frac{1}{2}\\
            &\frac{1}{3^p}+\frac{1}{4^p}>\frac{1}{2}\\
            &\frac{1}{5^p}+\frac{1}{6^p}+\frac{1}{7^p}+\frac{1}{8^p}>\frac{1}{2}\\
            &\cdots\cdots\\
            &\frac{1}{\left(2^k+1\right)^p}+\frac{1}{\left(2^k+2\right)^p}+\cdots+\frac{1}{\left(2^{k+1}\right)}>\frac{1}{2}
        \end{align*}于是\[
        a_{2^n}\geqslant1+\frac{n}{2}
        \]表明$0<p\leqslant 1$时$\left\{a_{2^n}\right\}$是正无穷大量且数列$\left\{a_n\right\}$单调增加即是正无穷大量.
    \end{Proof}
\end{brown}
\begin{red}
    \begin{remark}
        \cref{ex:2}中$p=1$即为调和级数.
    \end{remark}
\end{red}
\begin{brown}
    \begin{example}
        证明$\displaystyle\left\{b_n=1+\frac{1}{2}+\frac{1}{3}+\cdots+\frac{1}{n}-\ln n\right\}$收敛.
    \end{example}
    \begin{Proof}
        因为\[
        \left(
            1+\frac{1}{n}
        \right)^n<\rme<\left(
            1+\frac{1}{n}
        \right)^{n+1}
        \]取对数即\[
        \frac{1}{n+1}<\ln\frac{n+1}{n}<\frac{1}{n}
        \]于是\[
        b_{n+1}-b_n=\frac{1}{n+1}-\ln\frac{n+1}{n}<0
        \]而\[
        b_n>\ln\frac{2}{1}+\ln\frac{3}{2}+\cdots+\ln\frac{n+1}{n}-\ln n=\ln\left(n+1\right)-\ln n>0
        \]即有下界,故$\left\{b_n\right\}$收敛.其极限记作Euler常数\[
        \gamma=\lim_{n\to\infty}\left(
            1+\frac{1}{2}+\frac{1}{3}+\cdots+\frac{1}{n}-\ln n
        \right)\qedhere
        \]
    \end{Proof}
\end{brown}
\subsection{闭区间套定理}
\begin{formal}
    \begin{definition}[闭区间套的定义]\label{def:闭区间套的定义}
        若一系列闭区间$\displaystyle\left\{\left[a_n,b_n\right]\right\}$满足:\begin{enumerate}[label={\textup{(\arabic*)}}]
            \item $\displaystyle\left[a_{n+1},b_{n+1}\right]\subset\left[a_n,b_n\right],n=1,2,3,\cdots$
            \item $\displaystyle\lim_{n\to\infty}\left(b_n-a_n\right)=0$
        \end{enumerate}则称为一个闭区间套.
    \end{definition}
\end{formal}
\begin{formal}
    \begin{theorem}[闭区间套定理]\label{thm:闭区间套定理}
        如果$\displaystyle\left\{\left[a_n,b_n\right]\right\}$是一个闭区间套,则存在唯一的实数$\xi\in\left[a_n,b_n\right],n=1,2,3,\cdots$且$\displaystyle\lim_{n\to\infty}a_n=\lim_{n\to\infty}b_n=\xi$.
    \end{theorem}
    \begin{Proof}
        因为\[
        a_1\leqslant a_2\leqslant\cdots\leqslant a_{n-1}\leqslant a_n<b_n\leqslant b_{n-1}\leqslant\cdots\leqslant b_2\leqslant b_1
        \]即$\left\{a_n\right\},\left\{b_n\right\}$均单调有界即收敛.

        设$\displaystyle\lim_{n\to\infty}a_n=\xi\Longrightarrow\lim_{n\to\infty}b_n=\lim_{n\to\infty}\left(b_n-a_n\right)+\lim_{n\to\infty}a_n=\xi$,其中$\xi$根据\cref{thm:单调收敛定理}即是$\left\{a_n\right\}$作为集合的上确界同时是$\left\{b_n\right\}$作为集合的下确界.于是\[
        a_n\leqslant \xi\leqslant b_n,n=1,2,3,\cdots
        \]即$\xi$在所有区间内.另设$\xi'$也满足上述结论,由夹逼准则即得\[\xi'=
        \lim_{n\to\infty}a_n=\lim_{n\to\infty}b_n=\xi
        \]保证了唯一性.
    \end{Proof}
\end{formal}
\begin{red}
    \begin{remark}
        \cref{thm:闭区间套定理}中若全改为开区间则仍成立\[
        \lim_{n\to\infty}a_n=\lim_{n\to\infty}b_n=\xi
        \]但这个$\xi$未必在所有区间内.
    \end{remark}
\end{red}
\begin{formal}
    \begin{theorem}[实数集不可列]\label{thm:实数集不可列}
        实数集$\mathbb{R}$是不可列的.
    \end{theorem}\begin{Proof}
        考虑反证法,设实数集$\mathbb{R}$可列,即存在某一种排列规则使得\[
        \mathbb{R}=\left\{
            x_1,x_2,x_3,\cdots,x_n,\cdots
        \right\}
        \]取定区间$x_1\notin\left[a_1,b_1\right]$然后三等分取得\[
        \left[
            a_1,\frac{2a_1+b_1}{3}
        \right],\left[
            \frac{2a_1+b_1}{3},\frac{a_1+2b_1}{3}
        \right],\left[
            \frac{a_1+2b_1}{3},b_1
        \right]
        \]其中至少一个不含$x_2$,记作$\left[a_2,b_2\right]$;然后循环该过程即得到了一个闭区间套$\displaystyle\left\{\left[a_n,b_n\right]\right\}$,则根据\cref{thm:闭区间套定理}知存在唯一实数$\xi$属于所有闭区间,矛盾.

        因此,实数集$\mathbb{R}$是不可列的.
    \end{Proof}
\end{formal}
\subsection{子列}
\begin{formal}
    \begin{definition}[子列的定义]\label{def:子列的定义}
        设数列$\left\{x_n\right\}$和一串严格单调增加的正整数\[
        n_1<n_2<n_3<\cdots<n_k<n_{k+1}<\cdots
        \]则\[
        x_{n_1},x_{n_2},x_{n_3},\cdots,x_{n_k},\cdots
        \]这个数列$\left\{x_{n_k}\right\}$称为$\left\{x_n\right\}$的一个子列.
    \end{definition}
\end{formal}
\begin{red}
    \begin{remark}
        \cref{def:子列的定义}中$n_k$下标表示子列的第$k$项是原数列的第$n_k$项.并且一定有\[
        n_k\geqslant k,k=1,2,3,\cdots
        \]
    \end{remark}
\end{red}
\begin{formal}
    \begin{theorem}[子列收敛]\label{thm:子列收敛}
        设$\left\{x_n\right\}$的一个子列$\left\{x_{n_k}\right\}$,若$\displaystyle\lim_{n\to\infty}x_n=a$,则$\displaystyle\lim_{k\to\infty}x_{n_k}=a$.
    \end{theorem}
\end{formal}
\begin{green}
    \begin{corollary}[子列极限不同]\label{cor:子列极限不同}
        若$\left\{x_n\right\}$的两个子列$\left\{x_{n_k}^{\left(1\right)}\right\},\left\{x_{n_k}^{\left(2\right)}\right\}$收敛于不同的极限,则$\left\{x_n\right\}$发散.
    \end{corollary}
\end{green}
\begin{brown}
    \begin{example}
        证明数列$\displaystyle\left\{\sin\frac{n\pi}{4}\right\}$发散.
    \end{example}
    \begin{Proof}
        取$n_k^{\left(1\right)}=4k,n_k^{\left(2\right)}=8k+2$即可.
    \end{Proof}
\end{brown}
\subsection{Bolzano-Weierstrass定理}
\begin{formal}
    \begin{theorem}[Bolzano-Weierstrass定理]\label{thm:Bolzano-Weierstrass定理}
        有界数列必有收敛子列.
    \end{theorem}\begin{Proof}
        设$\left\{x_n\right\}$有界,设\[
        a_1\leqslant x_n\leqslant b_1,\forall n=1,2,3,\cdots
        \]等分为$\displaystyle\left[a_1,\frac{a_1+b_1}{2}\right],\left[\frac{a_1+b_1}{2},b_1\right]$,他们至少之一包含$\left\{x_n\right\}$的无穷多项,记作$\left[a_2,b_2\right]$,如此往复得到一个闭区间套$\left\{\left[a_k,b_k\right]\right\}$,他们都包含$\left\{x_n\right\}$的无穷多项.根据\cref{thm:闭区间套定理}知存在唯一实数$\xi$属于所有闭区间,且\[
        \lim_{k\to\infty}a_k=\lim_{k\to\infty}b_k=\xi
        \]按照\cref{def:子列的定义}取$x_{n_i}\in\left[a_i,b_i\right],i=1,2,3,\cdots$构成子列$\left\{x_{n_k}\right\}$并且\[
        a_k\leqslant x_{n_k}\leqslant b_k,k=1,2,3,\cdots
        \]夹逼得到\[
        \lim_{k\to\infty}x_{n_k}=\xi\qedhere
        \]
    \end{Proof}
\end{formal}
\begin{formal}
    \begin{theorem}[无界情形]\label{thm:无界情形}
        类似于\cref{thm:Bolzano-Weierstrass定理},若$\left\{x_n\right\}$无界,则必定存在一个子列$\left\{x_{n_k}\right\}$\[
        \lim_{k\to\infty}x_{n_k}=\infty
        \]
    \end{theorem}
    \begin{Proof}
        容易取得子列$\left\{x_{n_k}\right\}$使得\[
        \left|x_{n_k}\right|>k,k=1,2,3,\cdots\qedhere
        \]
    \end{Proof}
\end{formal}
\subsection{Cauchy收敛原理}
\begin{formal}
    \begin{definition}[基本数列的定义]\label{def:基本数列的定义}
        若数列$\left\{x_n\right\}$满足:$\forall \varepsilon>0,\exists N\in\mathbb{N}^+,\forall n,m>N\left(n>m>N\right):$\[
        \left|x_n-x_m\right|<\varepsilon
        \]则称$\left\{x_n\right\}$为基本数列.
    \end{definition}
\end{formal}
\begin{brown}
    \begin{example}
        证明数列$\displaystyle\left\{x_n=1+\frac{1}{2^2}+\frac{1}{3^2}+\cdots+\frac{1}{n^2}\right\}$是一个基本数列.
    \end{example}
    \begin{Proof}
        考虑$m>n$:\begin{align*}
            x_m-x_n&=\frac{1}{\left(n+1\right)^2}+\frac{1}{\left(n+2\right)^2}+\cdots+\frac{1}{m^2}\\
            &<\frac{1}{n\left(n+1\right)}+\frac{1}{\left(n+1\right)\left(n+2\right)}+\cdots+\frac{1}{\left(m-1\right)m}\\
            &=\frac{1}{n}-\frac{1}{m}<\frac{1}{n}\qedhere
        \end{align*}
    \end{Proof}
\end{brown}
\begin{brown}
    \begin{example}
        证明数列$\displaystyle\left\{x_n=1+\frac{1}{2}+\frac{1}{3}+\cdots+\frac{1}{n}\right\}$不是一个基本数列.
    \end{example}
    \begin{Proof}
        $\forall n\in\mathbb{N}^+:$\[
        x_{2n}-x_n=\frac{1}{n+1}+\frac{1}{n+2}+\cdots+\frac{1}{2n}>\frac{n}{2n}=\frac{1}{2}\qedhere
        \]
    \end{Proof}
\end{brown}
\begin{green}
    \begin{lemma}[基本数列有界]\label{lem:基本数列有界}
        基本数列必定有界.
    \end{lemma}
    \begin{Proof}
        取$\varepsilon_0=1$则$\exists N_0,\forall n>N_0:$\[
        \left|
            x_{N_0}-x_{N_0+1}
        \right|<1
        \]令$M=\max\,\left\{
            \left|x_1\right|,\left|x_2\right|,\cdots,\left|x_{N_0}\right|,1+\left|x_{N_0+1}\right|+1
        \right\}$于是$\forall n:\left|x_n\right|<M$证毕.
    \end{Proof}
\end{green}
\begin{formal}
    \begin{theorem}[Cauchy收敛原理/实数系的完备性]\label{thm:Cauchy收敛原理/实数系的完备性}
        数列$\left\{x_n\right\}$收敛当且仅当$\left\{x_n\right\}$是一个基本数列.
    \end{theorem}
    \begin{Proof}
        先证必要性.设$\left\{x_n\right\}$收敛于$a$,即$\forall\varepsilon>0,\exists N,\forall n,m>N:$\[
        \left|x_n-a\right|<\frac{\varepsilon}{2},\left|x_m-a\right|<\frac{\varepsilon}{2}
        \]于是\[
        \left|x_n-x_m\right|\leqslant\left|x_n-a\right|+\left|x_m-a\right|<\varepsilon
        \]

        再证充分性.因为$\left\{x_n\right\}$是基本数列,由\cref{lem:基本数列有界}知有界,由\cref{thm:Bolzano-Weierstrass定理}知存在收敛子列$\left\{x_{n_k}\right\}:$\[
        \lim_{k\to\infty}x_{n_k}=\xi
        \]又根据基本数列得到$\forall \varepsilon>0,\exists N,\forall n,m>N:$\[
        \left|
            x_n-x_m
        \right|<\frac{\varepsilon}{2}
        \]取$x_m=x_{n_k}$(其中$k$充分大使得$n_k>N$),取$k\to\infty$即得\[
        \left|
            x_n-\xi
        \right|\leqslant\frac{\varepsilon}{2}<\varepsilon
        \]于是该基本数列$\left\{x_n\right\}$必定收敛.
    \end{Proof}
\end{formal}
\begin{green}
    \begin{corollary}[实数系的完备性]\label{cor:实数系的完备性}
        实数构成的基本数列必定存在实数极限.
    \end{corollary}
\end{green}
\begin{red}
    \begin{remark}
        有理数集$\mathbb{Q}$不具有完备性,例如$\displaystyle\left\{\left(1+\frac{1}{n}\right)^n\right\}$是有理数构成的基本数列,但它的极限$\rme$是无理数(这一证明目前无法做到).
    \end{remark}
\end{red}
\begin{brown}
    \begin{example}
        证明:若数列$\left\{x_n\right\}$满足:\[
        \left|x_{n+1}-x_n\right|\leqslant\left|
            x_n-x_{n-1}
        \right|,0<k<1,n=2,3,\cdots
        \]则$\left\{x_n\right\}$收敛.
    \end{example}
    \begin{Proof}
        \begin{align*}
            \left|x_m-x_n\right|&\leqslant\left|x_m-x_{m-1}\right|+\left|x_{m-1}-x_{m-2}\right|+\cdots+\left|x_{n+1}-x_n\right|\\
            &\leqslant \left(k^{m-2}+k^{m-3}+\cdots+k^{n-1}\right)\left|x_2-x_1\right|\\
            &<\frac{k^{n-1}}{1-k}\left|x_2-x_1\right|\to0\left(n\to\infty\right)\qedhere
        \end{align*}
    \end{Proof}
\end{brown}
\subsection{实数系的基本定理}
\begin{red}
    \begin{remark}
        从\cref{thm:确界存在定理/实数系连续性定理}确界存在定理/实数系连续性定理$\Longrightarrow$\cref{thm:单调收敛定理}单调收敛定理$\Longrightarrow$\cref{thm:闭区间套定理}闭区间套定理$\Longrightarrow$\cref{thm:Bolzano-Weierstrass定理}\textup{Bolzano-Weierstrass}定理$\Longrightarrow$\cref{thm:Cauchy收敛原理/实数系的完备性}\textup{Cauchy}收敛原理/实数系的完备性这一推理过程可以看出,实数系的连续性包含了实数系的完备性.

        下面将证明反过来也是成立的.
    \end{remark}
\end{red}
\begin{formal}
    \begin{theorem}[实数系的连续性与完备性]\label{thm:实数系的连续性与完备性}
        实数系的完备性等价于实数系的连续性.
    \end{theorem}
    \begin{Proof}
        证明思路即:\cref{thm:Cauchy收敛原理/实数系的完备性}\textup{Cauchy}收敛原理/实数系的完备性$\Longrightarrow$\cref{thm:闭区间套定理}闭区间套定理$\Longrightarrow$\cref{thm:确界存在定理/实数系连续性定理}确界存在定理/实数系连续性定理.

        先证明第一步.设$\left\{\left[a_n,b_n\right]\right\}$是一系列闭区间且满足:\begin{enumerate}[label={\textup{(\arabic*)}}]
            \item $\left[a_{n+1},b_{n+1}\right]\subset\left[a_n,b_n\right],n=1,2,3,\cdots$
            \item $\displaystyle\lim_{n\to\infty}\left(b_n-a_n\right)=0$
        \end{enumerate}设$m>n$即得\[
        0\leqslant a_m-a_n<b_n-a_n\to0\left(
            n\to\infty
        \right)
        \]所以$\left\{a_n\right\}$是一个基本数列从而其收敛\[
        \lim_{n\to\infty}a_n=\xi
        \]那么\[
        \lim_{n\to\infty}b_n=\lim_{n\to\infty}\left(b_n-a_n\right)+\lim_{n\to\infty}a_n=\xi
        \]并考虑到$\left\{a_n\right\}$单调增加,$\left\{b_n\right\}$单调减少,于是$\xi$属于所有的闭区间.于是\cref{thm:闭区间套定理}闭区间套定理得证.

        再证第二步.设$S$是非空有上界的实数集合,设$T$为$S$的全体上界集合.取$a_1\notin T,b_1\in T\Longrightarrow a_1<b_1$于是$\displaystyle\left[a_1,\frac{a_1+b_1}{2}\right],\left[\frac{a_1+b_1}{2},b_1\right]$二者中,若$\displaystyle\frac{a_1+b_1}{2}\in T$,取$\displaystyle\left[a_2,b_2\right]=\left[a_1,\frac{a_1+b_1}{2}\right]$;若$\displaystyle\frac{a_1+b_1}{2}\notin T$,取$\displaystyle\left[a_2,b_2\right]=\left[\frac{a_1+b_1}{2},b_1\right]$.总之如此循环得到一个闭区间套$\left\{\left[a_n,b_n\right]\right\}$,其中\[
        a_n\notin T,b_n\in T,n=1,2,3,\cdots
        \]下面证明\cref{thm:闭区间套定理}得出的$\xi$是$T$的最小数即$S$的上确界.\begin{enumerate}[label={\textup{(\arabic*)}}]
            \item 一方面,若$\xi\notin T$,则$\exists x\in S,\xi<x$,于是$n$充分大时$b_n<x$,这与$b_n\in T$矛盾.于是$\xi\in T$.
            \item 另一方面,若$\exists \eta\in T,\eta<\xi$,于是$n$充分大时$a_n>\eta$,这与$a_n\notin T$矛盾.于是$\xi$是$T$的最小数.
        \end{enumerate}于是$\xi$是$T$的最小数即$S$的上确界.证毕.
    \end{Proof}
\end{formal}
\begin{red}
    \begin{remark}
       至此,我们目前在实数系下得到的\cref{thm:确界存在定理/实数系连续性定理}确界存在定理/实数系连续性定理,\cref{thm:单调收敛定理}单调收敛定理,\cref{thm:闭区间套定理}闭区间套定理,\cref{thm:Bolzano-Weierstrass定理}\textup{Bolzano-Weierstrass}定理,\cref{thm:Cauchy收敛原理/实数系的完备性}\textup{Cauchy}收敛原理/实数系的完备性这五大定理之间都是互相等价的,他们都可以称作实数系的基本定理.
    \end{remark}
\end{red}