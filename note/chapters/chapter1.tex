\newpage
\chapter{集合与映射}
\setcounter{section}{0}
\section{集合}
\subsection{集合}
\begin{formal}
\begin{definition}[集合的定义]\label{def:集合的定义}
    集合(集)是具有某种特定性质的具体的或抽象的对象汇集成的总体.这些对象称为集合的元素.
\end{definition}
\end{formal}
\begin{formal}
\begin{definition}[元素与集合的关系]\label{def:元素与集合的关系}
    如果$x$是集合$A$的一个元素,则称$x$属于集合$A$,记作$x\in A$;如果$x$不是集合$A$的一个元素,则称$x$不属于集合$A$,记作$x\notin A$或者$x\nin A$.
\end{definition}
\end{formal}
\begin{brown}
\begin{example}下面是一些常见数集的符号\[
\begin{matrix}
    \text{正整数集}&\text{自然数集}&\text{整数集}&\text{有理数集}&\text{实数集}&\text{复数集}\\
    \mathbb{N}^+&\mathbb{N}&\mathbb{Z}&\mathbb{Q}&\mathbb{R}&\mathbb{C}
\end{matrix}\]
\end{example}
\end{brown}
\begin{formal}
\begin{proposition}[集合的表示方法]\label{prop:集合的表示方法}
    集合的表示法有\begin{enumerate}[label={\textup{(\arabic*)}}]
        \item 枚举法:将集合内的元素一一列举出来,如光学三基色集合表示为$\left\{
            \text{红},\text{绿},\text{蓝}
        \right\}$,元素$a,b,c,d,e$组成的集合集合$\left\{
            a,b,c,d,e
        \right\}$,整数集\[
        \mathbb{Z}=\left\{
            0,\pm 1,\pm 2,\pm 3,\cdots,\pm n,\cdots
        \right\}
        \]
        \item 描述法:用集合元素的共同性质$P$来描述该集合\[
        S=\left\{
            x\mid x\text{满足性质}P
        \right\}
        \]如由$2$的平方根组成的集合$\left\{
            x\mid x^2=2
        \right\}$,有理数集\[
        \mathbb{Q}= \left\{
            \frac{q}{p}\mid q\in \mathbb{N}^+,p\in \mathbb{Z},p\neq 0
        \right\}
        \]
    \end{enumerate}
\end{proposition}
\end{formal}
\begin{formal}
\begin{theorem}[集合中元素的性质]\label{thm:集合中元素的性质}
    集合元素具有这样的性质\begin{enumerate}[label={\textup{(\arabic*)}}]
        \item 集合中的元素是无序的,即集合中的元素之间没有先后次序之分.
        \item 集合中的元素是互异的,即集合中的元素不重复.
        \item 集合中的元素是确定的,即集合中的元素是确定的.
    \end{enumerate}
\end{theorem}
\end{formal}
\begin{red}
\begin{remark}
    集合中的元素是确定的,不可能存在无法断言是否属于某一个集合的元素.
\end{remark}
\end{red}
\begin{formal}
\begin{definition}[空集的定义]\label{def:空集的定义}
    不含任何元素的集合称为空集,记作$\varnothing$或者$\left\{\right\}$.
\end{definition}
\end{formal}
\subsection{子集}
\begin{formal}
    \begin{definition}[子集的定义]\label{def:子集的定义}
        设集合$S,T$,若$\forall x\in S,$都有$x\in T$,则称$S$是$T$的子集,记作$S\subset T$或者$T\supset S$.若$\exists y\in S,y\nin T$则$S$不是$T$的子集,记作$S\not\subset T$.
    \end{definition}
\end{formal}
\begin{brown}
    \begin{example}
        \[
        \mathbb{N}\subset \mathbb{Z}\subset \mathbb{Q}\subset \mathbb{R}\subset \mathbb{C}
        \]
    \end{example}
\end{brown}
\begin{red}
    \begin{remark}
        显然地,对于任意一个集合$S$,有$S\subset S, \varnothing\subset S$.
    \end{remark}
\end{red}
\begin{formal}
\begin{definition}[真子集]\label{def:真子集}
    设$S\subset T$,若$\exists x\in T,x\nin S,$则称$S$是$T$的真子集,记作$S\subsetneq T$或者$T\supsetneq S$.
\end{definition}
\end{formal}
\begin{red}
    \begin{remark}
        集合$S$的元素个数一般记作$\left|S\right|$或$\#S$.
    \end{remark}
\end{red}
\begin{formal}
\begin{theorem}[子集数量]\label{thm:子集数量}
    设$S$是一个有限集合,则$S$的子集共有$2^{\left| S \right|}$个.
\end{theorem}
\begin{Proof}
$\forall x\in S$,都有$x\in S$或者$x\notin S$,故$S$的子集共有$2^{\left| S \right|}$个.
\end{Proof}
\end{formal}
\begin{formal}
\begin{definition}[区间的定义]\label{def:区间的定义}
    区间是实数集的子集,设$a<b\in \mathbb{R}$,则区间的定义为\[
    \begin{cases*}
           (a,b)=\left\{
                x\mid a<x<b
            \right\}\\
            [a,b]=\left\{
                x\mid a\leq x\leq b
            \right\}\\
            [a,b)=\left\{
                x\mid a\leq x<b
            \right\}\\
            (a,b]=\left\{
                x\mid a<x\leq b
            \right\}\\
            \left(
                a,+ \infty
            \right)  =\left\{
                x\mid x>a
            \right\}\\
            \left[
                a,+ \infty
            \right) =\left\{
                x\mid x\geqslant a
            \right\}\\
            \left(
                -\infty,b
            \right) =\left\{
                x\mid x<b
            \right\}\\
            \left(
                -\infty,b
            \right] =\left\{
                x\mid x\leqslant b
            \right\}\\
            \left(
                -\infty,+ \infty
            \right) =\left\{
                x\mid x\in \mathbb{R}
            \right\}=\mathbb{R}
    \end{cases*}
    \]
\end{definition}
\end{formal}
\subsection{运算}
\begin{formal}
    \begin{definition}[集合的相等]\label{def:集合的相等}
        设$S,T$是两个集合,若$S\subset T$且$T\subset S$,则称$S$等于$T$,记作$S=T$.
    \end{definition}
\end{formal}
\begin{formal}
    \begin{definition}[集合的并、交、差、补]\label{def:集合的并、交、差、补}
        集合的运算主要有并、交、差、补:
        \begin{enumerate}[label={\textup{(\arabic*)}}]
            \item 集合$S$与集合$T$的并指的是二者所有元素组成的集合\[
            S\cup T=\left\{
                x\mid x\in S\text{或}x\in T
            \right\}
            \]
            \item 集合$S$与集合$T$的交指的是二者共有元素组成的集合\[
            S\cap T=\left\{
                x\mid x\in S\text{且}x\in T
            \right\}
            \]
            \item 集合$S$与集合$T$的差指的是$S$中除去$T$中元素后的集合\[
             S-T=S\backslash T=\left\{
                x\mid x\in S\text{且}x\notin T
            \right\}
            \]
            \item 设$S\subset U$,则$S$关于$U$的补集定义为\[
            S^C_U=\complement_US=U-S=U\backslash S=\left\{
                x\mid x\in U\text{且}x\notin S
            \right\}
            \]在不混淆的前提下,也记作$\overline{S}$
        \end{enumerate}
    \end{definition}
\end{formal}
\begin{formal}
\begin{proposition}[集合的运算律]\label{prop:集合的运算律}
    设$S\subset U,T\subset U,X\subset US$则\begin{enumerate}[label={\textup{(\arabic*)}}]
        \item 交换律:\begin{align*}
            &S\cup T=T\cup S\\
            &S\cap T=T\cap S
        \end{align*}
        \item 结合律:\begin{align*}
            &S\cup \left( T\cup X \right)=\left( S\cup T \right)\cup X\\
            &S\cap \left( T\cap X \right)=\left( S\cap T \right)\cap X
        \end{align*}
        \item 分配律:\begin{align*}
            &S\cup \left( T\cap X \right)=\left( S\cup T \right)\cap \left( S\cup X \right)\\
            &S\cap \left( T\cup X \right)=\left( S\cap T \right)\cup \left( S\cap X \right)
        \end{align*}
        \item 补集的运算律:\begin{align*}
            &U\backslash S\cup S =U\\
            &U\backslash S\cap S=\varnothing
        \end{align*}\[
        S\backslash T =S\cap U\backslash T
        \]
        \item 对偶律(\textup{De Morgan}公式):\begin{align*}
            &\overline{S\cup T}=\overline{S}\cap \overline{T}\\
            &\overline{S\cap T}=\overline{S}\cup \overline{T}
        \end{align*}
    \end{enumerate}
\end{proposition}
\begin{Proof}
    \begin{enumerate}[label={\textup{(\arabic*)}}]
        \item 由定义易得.
        \item 由定义即得.
        \item 设$x\in S\cup\left(T\cap X\right)$即$x\in S$或者$x\in \left(T\cap X\right)\Longleftrightarrow x\in T,x\in X$,于是$x\in \left(S\cup T\right),x\in \left(S\cup X\right)$即$x\in \left(S\cup T\right)\cap \left(S\cup X\right)$于是$S\cup\left(T\cap X\right)\subset \left(S\cup T\right)\cap \left(S\cup X\right)$,同理可得$
        S\cup\left(T\cap X\right)\supset \left(S\cup T\right)\cap \left(S\cup X\right)$,故根据\cref{def:集合的相等}知$S\cup\left(T\cap X\right)=\left(S\cup T\right)\cap \left(S\cup X\right)$,同理可得第二个等式.
        \item 由定义即得.
        \item 设$x\in \overline{S\cup T}$即$x\nin S\cup T\Longleftrightarrow x\nin S,x\nin T$,于是$x\in \overline{S},x\in \overline{T}$即$x\in \overline{S}\cap \overline{T}\Longrightarrow
        \overline{S\cup T}\subset \overline{S}\cap \overline{T}$.同理可得$\overline{S\cup T}\supset \overline{S}\cap \overline{T}$,故根据\cref{def:集合的相等}知$\overline{S\cup T}=\overline{S}\cap \overline{T}$,同理可得第二个等式.
    \end{enumerate}
\end{Proof}
\end{formal}
\subsection{有限集与无限集}
\begin{formal}
\begin{definition}[有限集与无限集]\label{def:有限集与无限集}
    若$\#S=n\in \mathbb{N}$确定,则称$S$是有限集,否则称$S$是无限集.
\end{definition}
\end{formal}
\begin{formal}
\begin{definition}[无限集的可列与不可列]\label{def:无限集的可列与不可列}
    若无限集$S$的元素可以按照某一种规律排成一列(全部确定),则称该无限集是可列的.
\end{definition}
\end{formal}
\begin{formal}
\begin{theorem}[无限集与可列集]\label{thm:无限集与可列集}
    任意一个无限集一定包含可列子集,但无限集不一定是可列的.
\end{theorem}
\end{formal}
\begin{brown}
    \begin{example}
        整数集$\mathbb{Z}$是可列集\[
        \mathbb{Z}=\left\{
            0,1,-1,2,-2,3,-3,\cdots,n,-n,\cdots
        \right\}
        \]而实数集$\mathbb{R}$不是可列集.
    \end{example}
\end{brown}
\begin{formal}
\begin{theorem}[可列个可列集之并]\label{thm:可列个可列集之并}
    设$A_1,A_2,\cdot$为可列个可列集,则它们的并\[
    \bigcup_{n=1}^{\infty}A_n=A_1\cup A_2\cup \cdots \cup A_n\cup \cdots=\left\{
        x\mid \exists n\in \mathbb{N}^+,x\in A_n
    \right\}
    \]是可列集.
\end{theorem}
\begin{Proof}
    设$\forall n\in \mathbb{N}^+$\[
    A_n=\left\{
        x_{n1},x_{n2},\cdots,x_{nk},\cdots
    \right\}
    \]然后按如下对角线法则做排列\[
    \begin{matrix}
        A_1\\
        \\
        A_2\\
        \\
        A_3\\
        \\
        A_4\\
        \\
        \cdots
    \end{matrix}\qquad
    \begin{matrix}
        x_{11}&&x_{12}&&x_{13}&&x_{14}&&\cdots\\
        &\swarrow &&\swarrow &&\swarrow &&\swarrow &\\
        x_{21}&&x_{22}&&x_{23}&&x_{24}&&\cdots\\
        &\swarrow &&\swarrow &&\swarrow &&\swarrow &\\
        x_{31}&&x_{32}&&x_{33}&&x_{34}&&\cdots\\
        &\swarrow &&\swarrow &&\swarrow &&\swarrow &\\
        x_{41}&&x_{42}&&x_{43}&&x_{44}&&\cdots\\
        &\swarrow &&\swarrow &&\swarrow &&\swarrow &\\
        \cdots&&\cdots&&\cdots&&\cdots&&\cdots
    \end{matrix}
    \]这样的排序一定不会漏掉任何一个元素,然后去掉重复的元素即可.
\end{Proof}
\end{formal}
\begin{formal}
    \begin{theorem}[有理数集可列]\label{thm:有理数集可列}
        有理数集$\mathbb{Q}$是可列集.
    \end{theorem}
    \begin{Proof}
        由\cref{thm:可列个可列集之并},只要证$\left(0,1\right]$即可.因为$\left(0,1\right]$中的有理数可唯一地表示为$\displaystyle
        \frac{q}{p},q\leqslant p\in \mathbb{N}^+,\left(p,q\right)=1$.按照如下排列:
        \begin{align*}
            &p=1:x_{11}=1\\
            &p=2:x_{21}=\frac{1}{2}\\
            &p=3:x_{31}=\frac{1}{3},x_{32}=\frac{2}{3}\\
            &p=4:x_{41}=\frac{1}{4},x_{42}=\frac{3}{4}\\
            &\cdots\cdots
        \end{align*}一般地,$p=n\geqslant 1$时至多有$n-1$个元,记作$x_{n1},x_{n2},\cdots,x_{nk_n},1\leqslant k_n\leqslant n.$于是做排列即可.
    \end{Proof}
\end{formal}
\subsection{Descartes乘积集合}
\begin{formal}
    \begin{definition}[Descartes乘积集合的定义]\label{def:Descartes乘积集合的定义}
        设$A,B$两个集合,定义$A$与$B$的\textup{Descartes}乘积集合为\[
        A\times B=\left\{
            \left( x,y \right)\mid x\in A,y\in B
        \right\}
        \]
    \end{definition}
\end{formal}
\begin{brown}
    \begin{example}
        特别地,$\mathbb{R}^2=\mathbb{R}\times\mathbb{R}$为\textup{Descartes}平面直角坐标系;$\mathbb{R}^3=\mathbb{R}\times\mathbb{R}\times\mathbb{R}$为\textup{Descartes}空间直角坐标系.
    \end{example}
\end{brown}

\newpage
\section{映射与函数}
\subsection{映射}
\begin{formal}
\begin{definition}[映射的定义]\label{def:映射的定义}
    设集合$X,Y$,若按照某种规则$f,\forall x\in X,\exists y\in Y$(唯一)与之对应,则称$f$是$X$到$Y$的一个映射:\begin{align*}
        f:X&\longrightarrow Y\\
        x&\longmapsto y=f\left( x \right)
    \end{align*}
    其中$y=f\left(x\right)$称为$x$在映射$f$下的像,$x$称为$y$在映射$f$下的一个原像(逆像).集合$X$称为映射$f$的定义域,记作$D_f$;而$X$中元素$x$在映射$f$下的像$y$的集合称为映射$f$的值域,记作$R_f$即\[
    R_f=\left\{
        y\mid y\in Y,y=f\left( x \right),x\in X 
    \right\}
    \]
\end{definition}
\end{formal}
\begin{red}
\begin{remark}
    对于每一个$x\in X$,它在$f$下的像必须是唯一确定的.
\end{remark}
\end{red}
\begin{red}
\begin{remark}
    映射的定义域$D_f=X$,值域$R_f\subset Y$.但是逆像可以不具有唯一性.
\end{remark}
\end{red}
\begin{formal}
    \begin{proposition}[映射的构成]\label{prop:映射的构成}
        映射的构成要求的基本要素是\begin{enumerate}[label={\textup{(\arabic*)}}]
            \item 集合$X$即定义域$D_f=X$
            \item 集合$Y$即值域的最大范围$R_f\subset Y$
            \item 对应法则$f$使得$\forall x\in X,\exists y\in Y$(唯一)与之对应
        \end{enumerate}
    \end{proposition}
\end{formal}
\begin{formal}
    \begin{definition}[满射与单射的定义]\label{def:满射与单射的定义}
        设$f:X\longrightarrow Y$
        \begin{enumerate}[label={\textup{(\arabic*)}}]
            \item 若逆像具备唯一性即$\forall x_1\neq x_2\in X,f\left(x_1\right)\neq f\left(x_2\right)$,则称$f$是一个单射($1-1$)
            \item 若值域$R_f=Y$,则称$f$是一个满射(\textup{onto})
            \item 若$f$既是单射又是满射,则称$f$是一个双射(一一对应)
        \end{enumerate}
    \end{definition}
\end{formal}
\begin{formal}
    \begin{definition}[逆映射的定义]\label{def:逆映射的定义}
        设单射$f:X\longrightarrow Y$,根据\cref{def:满射与单射的定义}知逆像具有唯一性,于是对应法则\begin{align*}
        g:R_f&\longrightarrow X\\
        y&\longmapsto x(f\left(x\right)=y)
        \end{align*}是一个从$R_f$到$X$的映射,称为$f$的逆映射,记作$g=f^{-1}$.其中$D_{f^{-1}}=R_f,R_{f^{-1}}=D_f=X.$
    \end{definition}
\end{formal}
\begin{formal}
    \begin{definition}[复合映射的定义]\label{def:复合映射的定义}
        设映射\begin{align*}
            f:X&\longrightarrow U_1\\
            x&\longmapsto u=f\left( x \right)
        \end{align*}和\begin{align*}
            g:U_2&\longrightarrow Y\\
            u&\longmapsto y=g\left( u \right)
        \end{align*}若$R_f\subset U_1=D_g$,则定义符合映射\begin{align*}
            h=g\circ f:X&\longrightarrow Y\\
            x&\longmapsto y=g\left( f\left( x \right) \right)
        \end{align*}
    \end{definition}
\end{formal}
\begin{formal}
    \begin{theorem}[映射与逆映射复合]\label{thm:映射与逆映射复合}
        特别地\[
        f\circ f^{-1}\left(y\right)=y\in R_f,f^{-1}\circ f\left(x\right)=x\in D_f
        \]
    \end{theorem}
\end{formal}
\begin{red}
    \begin{remark}
        \cref{def:复合映射的定义}的关键是$R_f\subset D_g$.
    \end{remark}
\end{red}
\begin{red}
\begin{remark}
    一般地,$g\circ f$存在不代表$f\circ g$存在,即使二者均存在也不一定相等.
\end{remark}
\end{red}
\begin{formal}
    \begin{theorem}[符合映射结合律]\label{thm:符合映射结合律}
        一般地,$\left(h\circ g\right)\circ f=h\circ\left(g\circ f\right).$
    \end{theorem}
\end{formal}
\subsection{一元实函数}
\begin{formal}
    \begin{definition}[一元实函数的定义]\label{def:一元实函数的定义}
        取\cref{def:映射的定义}中$X\subset\mathbb{R},Y=\mathbb{R}$,则映射\begin{align*}
            f:X&\longrightarrow Y=\mathbb{R}\\
            x&\longmapsto y=f\left( x \right)
        \end{align*}称为一元实函数,简称函数,记作$y=f\left(x\right),x\in D_f.$
    \end{definition}
\end{formal}
\begin{formal}
    \begin{definition}[自变量、因变量与函数关系的定义]\label{def:自变量、因变量与函数关系的定义}
        函数$y=f\left(x\right)$中$x$称为自变量,$y$称为因变量,函数关系指的是自变量$x$与因变量$y$之间的对应关系.
    \end{definition}
\end{formal}
\subsection{基本初等函数}
\begin{formal}
    \begin{proposition}[基本初等函数]\label{prop:基本初等函数}
        $6$类基本初等函数是\begin{enumerate}[label={\textup{(\arabic*)}}]
            \item 常数函数:$y=c\left(c\in \mathbb{R}\right)$
            \item 幂函数:$y=x^{\alpha}\left(\alpha\in\mathbb{R}\right)$
            \item 指数函数:$y=a^x\left(a>0,a\neq 1\right)$
            \item 对数函数:$y=\log_a x\left(a>0,a\neq 1\right)$
            \item 三角函数:$y=\sin x,y=\cos x,y=\tan x$等
            \item 反三角函数:$y=\arcsin x,y=\arccos x,y=\arctan x$等
        \end{enumerate}
    \end{proposition}
\end{formal}
\begin{formal}
    \begin{definition}[初等函数的定义]\label{def:初等函数的定义}
        由基本初等函数经过有限次四则运算与复合得到的函数称为初等函数.
    \end{definition}
\end{formal}
\begin{formal}
    \begin{definition}[初等函数的自然定义域的定义]\label{def:初等函数的自然定义域的定义}
        初等函数的自然定义域指的是其自变量尽可能取到的最大取值范围.
    \end{definition}
\end{formal}
\subsection{函数的分段表示、隐式表示与参数表示}
\begin{formal}
    \begin{definition}[函数的分段表示的定义]\label{def:函数的分段表示的定义}
        设$A=D_{\varphi}\subset\mathbb{R},B=D_{\psi}\subset\mathbb{R},A\cap B=\varnothing,$则\[
        f\left(x\right)=\begin{cases*}
            \varphi\left(x\right),&$x\in A$\\
            \psi\left(x\right),&$x\in B$
        \end{cases*}
        \]是$A\cup B$上的分段函数.
    \end{definition}
\end{formal}
\begin{brown}
    \begin{example}
        符号函数\[
        \text{sgn}\left(x\right)=\begin{cases*}
            1,&$x>0$\\
            0,&$x=0$\\
            -1,&$x<0$
        \end{cases*}
        \]
    \end{example}
\end{brown}
\begin{formal}
    \begin{definition}[函数的隐式表示的定义]\label{def:函数的隐式表示的定义}
        通过方程\[
        F\left(x,y\right)=0
        \]确定的函数$y=f\left(x\right)$称为函数的隐式表示.
    \end{definition}
\end{formal}
\begin{brown}
    \begin{example}
        天体力学中的\textup{Kepler}方程通过\[
        y=x+\varepsilon\sin y,\varepsilon\in\left(0,1\right)
        \]确定.
    \end{example}
\end{brown}
\begin{formal}
    \begin{definition}[函数的参数表示的定义]\label{def:函数的参数表示的定义}
        引入参数$t$,通过方程组\[
        \begin{cases*}
            x=\varphi\left(t\right),\\
            y=\psi\left(t\right),
        \end{cases*}t\in\left[a,b\right]
        \]建立$t$与$x,y$的关系从而间接确定函数$y=f\left(x\right)$称为函数的参数表示.
    \end{definition}
\end{formal}
\begin{brown}
    \begin{example}
        旋轮线(摆线)的参数方程为\[
        \begin{cases*}
            x=a\left(t-\sin t\right),\\
            y=a\left(1-\cos t\right),
        \end{cases*}t\in\left[0,+\infty\right]
        \]
    \end{example}
\end{brown}
\subsection{函数的简单特性}
\begin{formal}
    \begin{definition}[函数的有界的定义]\label{def:函数的有界的定义}
        设$\exists m,M\in \mathbb{R}$使得$\forall x\in D$均有\[
        m\leqslant f\left(x\right)\leqslant M
        \]则称$f\left(x\right)$在$D$有界,其中$m$称为函数的下界,$M$称为函数的上界.
    \end{definition}
\end{formal}
\begin{red}
\begin{remark}
    有界函数的上下界不唯一.
\end{remark}
\end{red}
\begin{formal}
    \begin{definition}[函数有界的等价定义]\label{def:函数有界的等价定义}
        \cref{def:函数的有界的定义}的等价表述为$\exists M\in\mathbb{R}^+$使得$\forall x\in D$均有\[
        \left|f\left(x\right)\right|\leqslant M
        \]则$f\left(x\right)$在$D$有界.
    \end{definition}
\end{formal}
\begin{formal}
    \begin{definition}[确界的定义]\label{def:确界的定义}
        上界的最小值称为上确界,下界的最大值称为下确界.
    \end{definition}
\end{formal}
\begin{formal}
    \begin{definition}[单调性的定义]\label{def:单调性的定义}
        对于函数$y=f\left(x\right),x\in D$其单调性定义为\begin{enumerate}[label={\textup{(\arabic*)}}]
            \item 若$\forall x_1<x_2\in D$均有$f\left(x_1\right)\leqslant f\left(x_2\right)$(或$f\left(x_1\right)<f\left(x_2\right)$),则称函数$f\left(x\right)$在$D$上单调增加(或严格单调增加),记作$f\left(x\right)\uparrow$(或$f\left(x\right)$严格$\uparrow$)
            \item 若$\forall x_1<x_2\in D$均有$f\left(x_1\right)\geqslant f\left(x_2\right)$(或$f\left(x_1\right)>f\left(x_2\right)$),则称函数$f\left(x\right)$在$D$上单调减少(或严格单调减少),记作$f\left(x\right)\downarrow$(或$f\left(x\right)$严格$\downarrow$)
        \end{enumerate}
    \end{definition}
\end{formal}
\begin{formal}
    \begin{definition}[奇偶性的定义]\label{def:奇偶性的定义}
        设函数$y=f\left(x\right),x\in D$的定义域$D$关于原点对称,则\begin{enumerate}[label={\textup{(\arabic*)}}]
            \item 若$\forall x\in D,f\left(-x\right)=f\left(x\right)$,则称函数$f\left(x\right)$是偶函数.图像体现为关于$y$轴对称.
            \item 若$\forall x\in D,f\left(-x\right)=-f\left(x\right)$,则称函数$f\left(x\right)$是奇函数.图像体现为关于原点对称.
        \end{enumerate}
    \end{definition}
\end{formal}
\begin{formal}
    \begin{definition}[周期性的定义]\label{def:周期性的定义}
        若$\exists 0<T\in\mathbb{R}$使得$\forall x\in D$均有\[
        f\left(x+T\right)=f\left(x\right)
        \]则称函数$f\left(x\right)$是周期函数,其中$T$称为函数$f\left(x\right)$的周期.特别地,若$T$是满足周期性的最小正数,则称$T$为函数$f\left(x\right)$的最小正周期.
    \end{definition}
\end{formal}
\begin{brown}
    \begin{example}
        并不是每一个周期函数都有最小正周期,例如下面的\textup{Dirichlet}函数\[
        f\left(x\right)=\begin{cases*}
            1,&$x\in\mathbb{Q}$\\
            0,&$x\notin\mathbb{Q}$
        \end{cases*}
        \]
    \end{example}
\end{brown}
\begin{formal}
    \begin{theorem}[三角不等式]\label{thm:三角不等式}
        $\forall x,y\in\mathbb{R}$
        \[
        \left|
            \left|x\right|-\left|y\right|
        \right|\leqslant \left|
            x+y
        \right|\leqslant\left|x\right|+\left|y\right|
        \]
    \end{theorem}
\end{formal}
\begin{formal}
    \begin{theorem}[基本不等式链]\label{thm:基本不等式链}
        设$x_1,x_2,\cdots,x_n>0$,则\[
        \frac{n}{
            \frac{1}{
                x_1
            }+\frac{1}{
                x_2
            }+\cdots+\frac{1}{
                x_n
            }
        }\leqslant\sqrt[n]{
            x_1x_2\cdots x_n
        }\leqslant\frac{
            x_1+x_2+\cdots+x_n
        }{n}\leqslant
        \sqrt{
            \frac{
                x_1^2+x_2^2+\cdots+x_n^2
            }{n}
        }
        \]即\[
        \frac{n}{\sum\limits_{i=1}^{n}\frac{1}{x_i}}\leqslant\sqrt[n]{\prod\limits_{i=1}^{n}x_i}\leqslant\frac{\sum\limits_{i=1}^{n}x_i}{n}\leqslant\sqrt{\frac{\sum\limits_{i=1}^{n}x_i^2}{n}}\Longleftrightarrow
        H_n\leqslant G_n\leqslant A_n\leqslant Q_n
        \]
    \end{theorem}
    \begin{Proof}
        先证明$G_n\leqslant A_n$,首先$n=1,2$显然成立.考虑$n=2^k,k\in \mathbb{N}^+,$若$k=2$则\[
        \frac{x_1+x_2+x_3+x_4}{4}\geqslant\frac{2\sqrt{x_1x_2}+2\sqrt{x_3x_4}}{4}\geqslant\sqrt[4]{x_1x_2x_3x_4}
        \]于是$\forall k\in\mathbb{N}^+$,该不等式成立.再考虑$n\neq 2^k,k\in\mathbb{N}^+$的情况,取$l\in\mathbb{N}^+$使得$2^{l-1}<n<2^l$,作\[
        \overline{x}=\sqrt[n]{x_1x_2\cdots x_n}
        \]考虑\[
        \frac{
            x_1+x_2+\cdots+x_n+\left(2^l-n\right)\overline{x}
        }{2^l}\geqslant \sqrt[2^l]{
            x_1x_2\cdots x_n\overline{x}^{2^l-n}
        }=\overline{x}
        \]整理即得.并考虑对$\displaystyle
        \frac{1}{x_1},\frac{1}{x_2},\cdots,\frac{1}{x_n}$运用该不等式得到$H_n\leqslant G_n$.而$A_n\leqslant Q_n$平方显然.
    \end{Proof}
\end{formal}