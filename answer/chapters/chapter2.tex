\chapter{数列极限}
\section{实数系的连续性}
略.
\section{数列极限}
略.
\section{收敛准则}
略.
\section{补充习题}
1.设$a>b>c>0,$求极限:\[
(1)\lim_{n\to\infty}\left(\frac{a^n+b^n+c^n}{3}\right)^{\frac{1}{n}};(2)\lim_{n\to\infty}\left(\frac{a^{\frac{1}{n}}+b^{\frac{1}{n}}+c^{\frac{1}{n}}}{3}\right)^n
\]
\begin{solution}
    $(1)$\begin{align*}
        a=\lim_{n\to\infty}\left(\frac{a^n}{3}\right)^{\frac{1}{n}}<\lim_{n\to\infty}\left(\frac{a^n+b^n+c^n}{3}\right)^{\frac{1}{n}}<\lim_{n\to\infty}\left(\frac{3a^n}{3}\right)^{\frac{1}{n}}=a
    \end{align*}

    $(2)$\begin{align*}
        \lim_{n\to\infty}\left(\frac{a^{\frac{1}{n}}+b^{\frac{1}{n}}+c^{\frac{1}{n}}}{3}\right)^n&=\rme^{\lim\limits_{t\to0}\frac{\ln\left(\frac{a^t-1+b^t-1+c^t-1}{3}+1\right)}{t}}\\
        &=\sqrt[3]{abc}
    \end{align*}
\end{solution}

4.$\displaystyle a_1>0,a_{n+1}=a_n+\frac{1}{a_n}$,证明$\left\{a_n\right\}$是正无穷大量,并求$a,\alpha$使得$a_n,an^{\alpha}$等价.
\begin{solution}
    注意到$\left\{a_n\right\}$严格单调递增,设其有上界则必定收敛,注意到这不可能,于是$\left\{a_n\right\}$是正无穷大量.

    接着考虑$\left\{a_n^2\right\}$,显然是严格单调递增的正无穷大量,因为\[
    a_{n+1}^2-a_n^2=2+\frac{1}{a_n^2}\to2
    \]那么考虑\textup{stolz}定理\[
    \lim_{n\to\infty}\frac{a_n^2}{n}=\lim_{n\to\infty}\frac{a_{n+1}^2-a_n^2}{1}=2\]那么$\displaystyle a=\sqrt{2},\alpha=\frac{1}{2}$.
\end{solution}

5.设$\displaystyle -1<a_0<1,a_n=\sqrt{\frac{1+a_{n-1}}{2}}\left(n>0\right)$,求$\displaystyle\lim_{n\to\infty}4^n\left(1-a_n\right)$.
\begin{solution}
    设$a_0=\cos\theta$,则$\displaystyle a_n=\cos\frac{\theta}{2^n}$,于是\begin{align*}
        \lim_{n\to\infty}4^n\left(1-a_n\right)&=\lim_{n\to\infty}4^n\left(1-\cos\frac{\theta}{2^n}\right)\\
        &=\lim_{n\to\infty}4^n\left(2\sin^2\frac{\theta}{2^{n+1}}\right)\\
        &=\lim_{n\to\infty}2^{2n+1}\sin^2\frac{\theta}{2^{n+1}}\\
        &=\frac{\theta^2}{2}\\
        &=\frac{\arccos^2a_0}{2}
    \end{align*}
\end{solution}

6.设$\displaystyle x_n=1+\frac{1}{\sqrt{2}}+\frac{1}{\sqrt{3}}+\cdots+\frac{1}{\sqrt{n}}-2\sqrt{n}$\begin{enumerate}[label={\textup{(\arabic*)}}]
    \item 证明$\displaystyle\left\{x_n\right\}$收敛
    \item 求$\displaystyle\lim_{n\to\infty}\left(\frac{1}{\sqrt{n}}+\frac{1}{\sqrt{2n}}+\cdots+\frac{1}{\sqrt{n^2}}\right)$
\end{enumerate}
\begin{solution}
    注意到\begin{align*}
        x_n&=\left(
            x_{n}-x_{n-1}
        \right)+\left(
            x_{n-1}-x_{n-2}
        \right)+\cdots+\left(
            x_2-x_1
        \right)+x_1\\
        &=\sum_{k=1}^{n-1}\left[
            \frac{1}{\sqrt{k+1}}-2\left(
                \sqrt{k+1}-\sqrt{k}
            \right)
        \right]-1\\
        &=-\sum_{k=1}^{n-1}\frac{1}{\sqrt{k+1}\left(\sqrt{k}+\sqrt{k+1}\right)^2}-1
    \end{align*}收敛.

    注意到\begin{align*}
        \lim_{n\to\infty}\left(
            \frac{1}{\sqrt{n}}+\frac{1}{\sqrt{2n}}+\cdots+\frac{1}{\sqrt{n^2}}
        \right)&=\lim_{n\to\infty}\frac{1}{\sqrt{n}}\left(
            1+\frac{1}{\sqrt{2}}+\cdots+\frac{1}{\sqrt{n}}
        \right)\\
        &=\lim_{n\to\infty}\frac{1}{\sqrt{n}}\left(x_n+2\sqrt{n}\right)\\
        &=2
    \end{align*}
\end{solution}

7.设单调增加的收敛数列$\left\{x_n\right\},\forall n\geqslant 2:\displaystyle\frac{x_{n+1}+x_{n-1}}{2}\leqslant x_{n}$,证明:$\displaystyle\lim_{n\to\infty}n\left(x_n-x_{n-1}\right)=0$
\begin{Proof}
    容易知道$\left\{x_n-x_{n-1}\right\}$是单调递减且非负的,设极限为$a$.设$\displaystyle\lim_{n\to\infty}n\left(x_{n}-x_{n-1}\right)>0$,因为$\exists N,\forall n>N:\displaystyle x_n-x_{n-1}>\frac{a}{2}$于是\[
    x_n\to\infty\left(
        n\to\infty
    \right)
    \]矛盾.
\end{Proof}

8.设非负数列$\displaystyle\left\{x_n\right\}:x_{n+1}\leqslant x_n+\frac{1}{n^2}$,证明$\left\{x_n\right\}$收敛.
\begin{Proof}
    考虑到\[
    x_{n+1}\leqslant x_n+\frac{1}{n^2}\leqslant x_n+\frac{1}{n-1}-\frac{1}{n}
    \]于是$\displaystyle\left\{y_n=x_n+\frac{1}{n-1}\right\}$单调递减且有下界,其收敛又$\displaystyle\lim_{n\to\infty}\frac{1}{n-1}=0$于是$\displaystyle\left\{x_n\right\}$收敛.
\end{Proof}

9.是否存在数列$\left\{x_n\right\}:\forall A\in\left[0,1\right],\exists\left\{x_{n_k}\right\}:\lim_{k\to\infty}x_{n_k}=A$.
\begin{Proof}
    这表明数列在$\left[0,1\right]$处处稠密.考虑$x_n=\sin n$.
\end{Proof}

10.\begin{enumerate}[label={\textup{(\arabic*)}}]
    \item 设$\displaystyle\left\{x_n\right\}:\forall \varepsilon>0,\exists N,\forall n>N:\left|x_{2n}-x_n\right|<\varepsilon$,则$\left\{x_n\right\}$ 是否收敛.
    \item 设$\displaystyle\left\{x_n\right\}:\forall\varepsilon>0,p\in\mathbb{N}^+,\exists N=N\left(\varepsilon,p\right),\forall n>N:\left|x_{n+p}-x_n\right|<\varepsilon$,则$\left\{x_n\right\}$是否收敛.
\end{enumerate}
\begin{solution}
    \begin{enumerate}[label={\textup{(\arabic*)}}]
        \item 否.考虑$\displaystyle x_n=\log_2\log_2 n$
        \item 否.考虑$\displaystyle x_n=\ln n$
    \end{enumerate}
\end{solution}

11.设$\displaystyle\lim_{n\to\infty}a_n=a,p_n>0,\lim_{n\to\infty}\frac{p_n}{p_1+p_2+\cdots+p_n}=0$,证明:\[
\lim_{n\to\infty}\frac{p_na_1+p_{n-1}a_2+\cdots+p_1a_n}{p_1+p_2+\cdots+p_n}=a
\]
\begin{Proof}
    设$\displaystyle\lim_{n\to\infty}a_n=a=0,\left|a_n\right|<M$,那么$\displaystyle\forall\varepsilon>0,\exists N_1,\forall n>N_1:\left|a_n\right|<\frac{\varepsilon}{2};\exists N_2,\forall n>N_2:\frac{p_n}{p_1+p_2+\cdots+p_n}<\frac{\varepsilon}{2MN_1}$,于是取$N=N_1+N_2,\forall n>N=N_1+N_2:$\begin{align*}
        \left|
            \frac{p_na_1+p_{n-1}a_2+\cdots+p_1a_n}{p_1+p_2+\cdots+p_n}
        \right|&\leqslant\left|
            \frac{p_1a_n+\cdots+p_{n-N_1}a_{N_1+1}}{p_1+\cdots+p_n}
        \right|+\left|
            \frac{p_na_1+\cdots+p_{n-N_1+1}a_{N_1}}{p_1+\cdots+p_n}
        \right|\\
        &\leqslant\frac{\varepsilon}{2}\left(
            \frac{p_1+\cdots+p_{n-N_1}}{p_1+\cdots+p_n}
        \right)+\frac{\varepsilon}{2MN_1}\cdot M\cdot N_1\\
        &=\varepsilon
    \end{align*}

    当$a\neq0$,考虑\[
    \left|
        \frac{p_na_1+p_{n-1}a_2+\cdots+p_1a_n}{p_1+p_2+\cdots+p_n}-a
    \right|=\left|
        \frac{
            p_n\left(a_1-a\right)+p_{n-1}\left(a_2-a\right)+\cdots+p_1\left(a_n-a\right)
        }{
            p_1+p_2+\cdots+p_n
        }
    \right|\qedhere
    \]
\end{Proof}