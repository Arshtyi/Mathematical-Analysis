\newpage
\chapter{集合与映射}
\section{集合}
略
\section{映射与函数}
13.求定义在$\left[0,1\right]$上的函数,是$\left[0,1\right]$到$\left[0,1\right]$的一一对应但在任一子区间不单调.
\begin{solution}
所求为\[
f\left(x\right)=\begin{cases*}
    x,&$x\in\mathbb{Q}$\\
    1-x,&$x\notin\mathbb{Q}$
\end{cases*}
\]
\end{solution}
\section{补充习题}
4.设$a,b\in\mathbb{R},$证明:\[
\frac{
    \left|a+b\right|
}{1+\left|a+b\right|}\leqslant\frac{\left|a\right|}{1+\left|a\right|}+\frac{\left|b\right|}{1+\left|b\right|}
\]\begin{Proof}
    考虑函数$\displaystyle f\left(x\right)=\frac{x}{1+x},x\geqslant 0$显然单调递增,由三角不等式知$0\leqslant \left|a+b\right|\leqslant\left|a\right|+\left|b\right|$,于是\begin{align*}
        \frac{
            \left|a+b\right|
        }{1+\left|a+b\right|}&\leqslant\frac{\left|a\right|+\left|b\right|}{
            1+\left|a\right|+\left|b\right|
        }\\
        &\leqslant
        \frac{\left|a\right|}{1+\left|a\right|}+\frac{\left|b\right|}{1+\left|b\right|}\qedhere
    \end{align*}
\end{Proof}
5.设$1<a\in\mathbb{R},2\leqslant n\in\mathbb{N}$,证明:\[
\frac{n}{a^n}<\frac{2}{\left(n-1\right)\left(a-1\right)^2}
\]
\begin{Proof}
    考虑函数\[
    f\left(x\right)=\frac{x^n}{n}-\frac{
        \left(n-1\right)\left(x-1\right)^2
    }{2},x>1,n\geqslant 2
    \]于是导数\begin{align*}
        f'\left(x\right)&=\left(1+x-1\right)^{n-1}-\left(n-1\right)\left(x-1\right)\\
        &\geqslant 1+\left(n-1\right)\left(x-1\right)-\left(n-1\right)\left(x-1\right)\\
        &=1>0\qedhere
    \end{align*}
\end{Proof}
6.证明:\begin{enumerate}[label={\textup{(\arabic*)}}]
    \item $\displaystyle
    \frac{1}{2\sqrt{n}}\leqslant \frac{1}{2}\cdot\frac{3}{4}\cdot\frac{5}{6}\cdots\frac{2n-1}{2n}<\frac{1}{\sqrt{2n+1}}
    $
    \item $\displaystyle
    \left(
        1+\frac{1}{2}+\cdots+\frac{1}{n}
    \right)^2<n\left(
        1+\frac{1}{2^2}+\cdots+\frac{1}{n^2}
    \right)
    $
\end{enumerate}
\begin{Proof}
    \begin{enumerate}[label={\textup{(\arabic*)}}]
        \item 记$\displaystyle S=\frac{1}{2}\cdot\frac{3}{4}\cdot\frac{5}{6}\cdots\frac{2n-1}{2n}$则\begin{align*}
            S^2&< \frac{1}{2}\cdot\frac{2}{3}\cdot\frac{3}{4}\cdot\frac{4}{5}\cdots\frac{2n-1}{2n}\cdot\frac{2n}{2n+1}\\
            &=\frac{1}{2n+1}
        \end{align*}又\begin{align*}
            S^2&\geqslant \frac{1}{4}\cdot\frac{2}{3}\cdot\frac{3}{4}\cdot\frac{4}{5}\cdot\frac{5}{6}\cdots\frac{2n-2}{2n-1}\cdot\frac{2n-1}{2n}\\
            &=\frac{1}{4n}
        \end{align*}于是\[
        \frac{1}{2\sqrt{n}}\leqslant \frac{1}{2}\cdot\frac{3}{4}\cdot\frac{5}{6}\cdots\frac{2n-1}{2n}<\frac{1}{\sqrt{2n+1}}
        \]当且仅当$n=1$时等号成立.
        \item 考虑Cauchy不等式\begin{align*}LHS=
            \left(
                1\cdot 1+1\cdot\frac{1}{2}+\cdots+1\cdot\frac{1}{n}
            \right)^2&< \left(
                1^2+1^2+\cdots+1^2
            \right)\left(
                1^2+\left(\frac{1}{2}\right)^2+\cdots+\left(\frac{1}{n}\right)^2
            \right)\\
            &=n\left(
                1+\frac{1}{2^2}+\cdots+\frac{1}{n^2}
            \right)=RHS\qedhere
        \end{align*}
    \end{enumerate}
\end{Proof}
8.构造如下一一映射:\begin{enumerate}[label={\textup{(\arabic*)}}]
    \item $\left(a,b\right)\longrightarrow\left[a,b\right]$
    \item $\mathbb{R}\longrightarrow\mathbb{Q}^C$
\end{enumerate}
\begin{solution}
    \begin{enumerate}[label={\textup{(\arabic*)}}]
        \item 记$\left(a,b\right)$中的有理数全体为$\left\{x_n\right\}$,构造$\left[a,b\right]\longrightarrow\left(a,b\right)$的一一对应为:\begin{align*}
            a&\longmapsto x_1\\
            b&\longmapsto x_2\\
            x_n&\longmapsto x_{n+2}
        \end{align*}其余无理数映射到自身.
        \item 考虑取一个简单的无理数例如$\sqrt{2}$,对于$q\in\mathbb{Q}$,映射到$q+\sqrt{2}$,对于形如$q+k\sqrt{2}\left(k\in\mathbb{N}\right)$的实数,映射到$q+\left(
            k+1
        \right)\sqrt{2}$
    \end{enumerate}
\end{solution}